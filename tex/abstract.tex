\begin{abstract}

Natural evolutionary history is profoundly shaped by evolutionary transitions in individuality, episodes where independent replicating entities united to form more complex replicating entities.
Fraternal transitions in individuality occur specifically when the derived replicating entity is composed of kin groups of a lower-level replicator;
examples include the evolution of multicellularity and the evolution of eusocial insect colonies.
The conditions necessary for fraternal transitions to occur and the mechanisms by which such transitions occur have been a fruitful target of scientific interest and open questions that warrant continued investigation remain.
We present an extension of previous exploratory work with the DISHTINY (Distributed Hierarchical Transitions in IndividualitY) framework.
In this current work we introduce a more open-ended digital organism model in which cells are controlled by heritable computer programs based on the SignalGP genetic programming representation, which allows for dynamic (event-driven) interactions among digital cells and between cells and their environment.
DISHTINY allows SignalGP-controlled cells to unite into explicitly registered kin cooperating groups.
We observe cells evolve phenotypes characteristic of fraternal transitions in individuality with respect to these cooperating groups: reproductive division of labor, resource sharing (including, in some replicates, endowment of offspring propagule groups), asymmetrical within-group and inter-group phenomena mediated by cell-cell messaging, morphological patterning, gene-regulation mediated life cycles, and adaptive apoptosis.
From an applied perspective, fraternal transitions in individuality might yield more complex digital organisms that exhibit more sophisticated, and potentially useful, capabilities.
As an experimental system, open-ended digital multicellular systems promise to provide unique insight into evolutionary transitions in individuality by allowing pursuit of such biologically-motivated questions that otherwise, \textit{in vivo}, might require experiments that are slow, expensive, or call for manipulation of physically immutable variables.

\end{abstract}
