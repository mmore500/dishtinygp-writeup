\begin{abstract}

Evolutionary transitions in individuality have profoundly shaped natural evolutionary history.
These transitions include episodes where independent replicating entities united to form more complex replicating entities.
If the derived replicating entity is composed of lower-level replicating entity kin, then such a transition is termed fraternal.
Examples of fraternal transitions include the evolution of multicellularity and the evolution of eusocial insect colonies.
The conditions necessary for fraternal transitions to arise and the mechanisms by which such transitions take place continue to be fruitful targets of scientific interest.
We work with replicators controlled by heritable open-ended event-driven computer programs.
Replicators were allowed to unite into explicitly registered kin cooperating groups.
We observe cells evolve phenotypes characteristic of fraternal transitions in individuality with respect to these cooperating groups: reproductive division of labor, resource sharing (including, in some treatments, endowment of offspring propagule groups), asymmetrical within-group and inter-group phenomena mediated by cell-cell messaging, morphological patterning, gene-regulation mediated life cycles, and adaptive apoptosis.
From an applied perspective, fraternal transitions in individuality might yield digital organisms that exhibit more sophisticated, and potentially useful, capabilities.
As an experimental system, open-ended digital multicellular systems allow pursuit of biologically-motivated questions that otherwise, \textit{in vivo}, might require experiments that are slow, expensive, or call for physically impracticable manipulations.

\end{abstract}
