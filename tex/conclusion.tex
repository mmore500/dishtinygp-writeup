\section{Conclusion}

In previous work exploring the DISHTINY platform, we used simple organisms that evolve parameters for a set of manually-designed strategies to demonstrate that DISHTINY selects for genotypes that exhibit high-level individuality \cite{moreno2019toward}.
In this work, we used digital organisms controlled by event-driven genetic programs capable of arbitrary computations.

We have detected phenotypic outcomes characteristic of fraternal transitions in individuality.
Across treatments, we observed resource-sharing and reproductive cooperation among same-channel groups.
These outcomes arose even in treatments where same-channel kin groups lacked functional significance (i.e., resource was distributed evenly), suggesting that reliable kin recognition alone is sufficient to observe aspects of fraternal collectivism evolve in systems where population members compete antagonistically for limited space or resources and spatial mixing is low.
In addition to their functional consequences, perhaps the role of physical mechanisms such as cell attachment simply as a kin recognition tool might merit consideration.

In individual replicates, we observed myriad evolved strategies exhibiting intercellular communication, coordination, and differentiation: endowment of offspring propagule groups, asymmetrical intra-group resource sharing, asymmetrical inter-group relationships, morphological patterning, gene-regulation mediated life cycles, and adaptive apoptosis.

We see several fruitful opportunities to manifest greater collective sophistication among evolved agents.
The modular design of SignalGP lends itself to the possibility of exploring sexual recombination.
DISHTINY's spatial, distributed nature also affords a route to achieve large-scale digital multicellularity experiments consisting of millions, instead of thousands, of cells via high-performance parallel computing.
We are interested in the possibility of allowing cell groups to develop neural and vascular networks in such work.
We hypothesize that selective pressures related to intra-group coordination and inter-group conflict might spur developmental and structural infrastructure that could be co-opted to evolve agents proficient at unrelated tasks like maze-solving, game-playing, or reinforcement learning.

As we develop and demonstrate DISHTINY, we are eager to undertake experiments investigating open questions pertaining to major evolutionary transitions such as the role of pre-existing phenotypic plasticity \citep{clune2007investigating, lalejini2016evolutionary}, pre-existing environmental interactions, pre-existing reproductive division of labor, and how transitions relate to increases in organizational \citep{goldsby2012task}, structural, and functional \citep{goldsby2014evolutionary} complexity.
