\section{Conclusion}

In previous work exploring the DISHTINY platform, we used simple organisms that evolve parameters for a set of manually-designed strategies to demonstrate that DISHTINY selects for genotypes that exhibit high-level individuality \cite{moreno2018toward}.
In this work, we used digital organisms controlled by genetic programs capable of arbitrary computations.
We have shown that DISHTINY induces detectable fraternal transitions in individuality on these more dynamic evolutionary substrate.

% We observed cell-, zeroth-, and first- level individuality among evolutionary outcomes.
% Specifically, we observed
% \begin{enumerate}
%   \item reproductive division of labor among members of the same channel (i.e. individuals enveloped in a same-channel signaling network ceded reproduction to those at the periphery), and
%   \item cooperation between members of the same channel (i.e. pooling of resource on same-channel signaling networks).
% \end{enumerate}
%
% Ecological trials revealed that first-level individuals usually outcompete zeroth- and cell-level individuals.
% We observed suppression of somatic mutation through apoptosis correlated with first-level individuality.
% The magnitude of resource endowment for propagules was also correlated with first-level individuality.
%
% Although shifts in individuality coincident with level-zero and level-one signaling networks were both clearly observed, the question of whether these transitions were truly hierarchical in nature is debatable.
% That is, it is not clear whether level-zero individuality was to some extent preserved in or necessary for the emergence of level-one individuality.
% Given the nature of the manually-designed strategies for resource-pooling and reproductive division of labor, level-one resource pooling and division of labor could readily leapfrog over level-zero resource pooling and division of labor and, in many ways, seemed to completely supersede those level-zero efforts.
%
% We believe that this is a shortcoming of the design of the simple cell-like organism employed in these experiments, not the DISHTINY platform itself.
% We have nevertheless clearly demonstrated that DISHTINY ultimately selects for high-level individuality.
We are eager to apply DISHTINY to scientific questions relating to major evolutionary transitions such as the role of pre-existing phenotypic plasticity \citep{clune2007investigating, lalejini2016evolutionary}, pre-existing environmental interactions, pre-existing reproductive division of labor, and how transitions relate to increases in organizational \citep{goldsby2012task}, structural, and functional \citep{goldsby2014evolutionary} complexity.

We believe that such an approach also provides a unique opportunity to fundamentally advance Artificial life with respect to open-ended evolution.
Fundamental to this goal is scale.
The DISHTINY platform trivially scales to select for an arbitrary number of hierarchical levels of individuality (not just the two hierarchical levels explored in these experiments).
Existing artificial life systems to study transitions in evolution where ``multicells'' exist in explicitly segregated sub-spaces --- such as \citep{goldsby2014evolutionary} --- can also scale to an arbitrary number of hierarchical levels by simply hooking the outputs of a multicell into the cell-level controls of a higher level multicell.
However, under our approach --- which unfolds within a single united spatial domain --- the granularity of the interface between individuals (and the interface between sub-units within individuals) scales with the cell count of individuals.
With respect to open-endedness, this property provides a unique opportunity: as higher levels of individuality evolve, the dynamics within and between individuals fundamentally differ at each level.

Importantly, the united spatial domain underpinning DISHTINY is designed in a decentralized manner and can comfortably scale as additional computing resources are provided.
Parallel computing is widely exploited in evolutionary computing, where subpopulations are farmed out for periods of isolated evolution or single genotypes are farmed out for fitness evaluation
\citep{lin1994coarse, real17a}.
DISHTINY presents a more fundamental parallelization potential: principled parallelization of the evolving individual phenotype at arbitrary scale (i.e. a high-level individual as a large collection of individual cells on the toroidal grid).
Such parallelization will be key to realizing evolving computational systems with scale --- and, perhaps, complexity --- approaching those of biological systems.
Such large-scale artificial life systems lend themselves to practical applications.
We plan to assay the potential of large-scale digital multicellularity as a substrate to evolve reinforcement-learning agents with an eye towards tasks pursued in existing and emerging AI research, such as the Open AI Gym and the Animal AI Olympics \citep{brockman2016openai}.
