\section{Introduction}

An evolutionary transition in individuality is an event where independently replicating entities unify to replicate as a higher-level individual \citep{smith1997major}.
These transitions are understood as essential to natural history's remarkable record of complexification and diversification \citep{smith1997major}.
Likewise, researchers studying open-ended evolution in artificial life systems have focused on transitions in individuality as a mechanism for driving the evolution of complexity and diversity \citep{taylor2016open, banzhaf2016defining}.

Here, we focus on \textit{fraternal} evolutionary transitions in individuality, in which the higher-level replicating entity is derived from the combination of cooperating kin that have entwined their long-term fates \citep{west2015major}.
Multicellular organisms and eusocial insect colonies exemplify this phenomenon \citep{smith1997major}.
Both are sustained and propagated through the cooperation of lower-level kin.
Although not our focus here, egalitarian transitions --- events in which non-kin unite, such as the genesis of mitochondria by symbiosis of free-living prokaryotes and eukaryotes \citep{smith1997major} --- also constitute essential episodes in natural history.

Because in nature major transitions typically occur rarely and over vast time scales, they are challenging to study.
Recent work in experimental evolution \citep{ratcliff2014experimental, ratcliff2015origins, gulli2019evolution, koschwanez2013improved}, mechanistic modeling \citep{hanschen2015evolutionary, staps2019emergence}, and artificial life \citep{goldsby2012task, goldsby2014evolutionary} complements traditional post hoc approaches focused on characterizing the record of natural history.
Made possible by these systems' instantiative nature, experimental manipulations affecting the evolutionary transition process itself shed special light on the requisites, mechanisms, and evolutionary consequences of fraternal transitions.

Our work here follows closely in the intellectual vein of Goldsby's deme-based artificial life experiments.
In this string of studies, spatially-segregated pockets of cells (``demes'') compete for space in a fixed-size population of demes.
Individual cells are controlled by self-replicating Avida-style computer programs with special instructions that allow them to interact with their environment and with neighboring cells.
Two modes of reproduction are defined under the deme model: within-deme and deme-founding.
In the first, a cell copies itself into a neighboring toroidal tile within its deme.
In the second, a deme slot is cleared in the deme population then seeded with a single cell from the parent deme.
Cells grow freely within demes, but deme fecundity depends on the collective profile of computational tasks (e.g., logic functions) performed within the deme.
This setup mirrors the dynamics of biological multicellularity, in which cell proliferation may either grow an existing multicellular body or spawn a new multicellular body.

However, we pursue a unified spatial model in which higher-order individuality emerges implicitly in terms of the actions of individual cells rather than explicitly in terms of an algorithmic distinction between two spatial scales.
In our view, the merit and significance of such a distinction on purely philosophical grounds is dubious.
However, more concretely, we hope that a unified spatial approach will enable more nuanced interactions between emergent individuals, which may interface in terms of many immediate cell-cell interfaces instead of a single deme-deme interface.
We conjecture that cell-cell interfaces between emergent units of individuality may promote the continued evolution of novelty in terms of interactions between competitors, interactions between propagule and parent groups, and, if nested hierarchical individuality were to emerge, fundamentally different building blocks at each level.

To realize fraternal transitions of individuality within a unified spatial realm, we reward explicitly-registered hereditary groups for cooperation on a distributed resource collection task.
Here, we extend previous work exploring the selective implications of this scheme by evolving parameters for manually-designed cell-level strategies \citep{moreno2019toward}.

Our work employs an event-driven genetic programming representation called SignalGP which was designed to facilitate dynamic interactions among agents and between agents and their environment \citep{lalejini2018evolving}.
We also debut a regulatory extension to the existing SignalGP system and a novel scheme to interface evolving agents with grid-based sensors and actuators which is designed to gird phenotypic cardinal symmetry.
