\section{Introduction}

An evolutionary transition in individuality is an event where independently replicating entities unify to replicate as a higher-level individual \citep{smith1997major}.
These transitions are understood as essential to natural history's remarkable record of complexification and diversification \citep{smith1997major}.
Likewise, researchers studying open-ended evolution in artificial life systems have focused on transitions in individuality as a mechanism for driving the evolution of complexity and diversity \citep{taylor2016open, banzhaf2016defining}.

Here, we focus on \textit{fraternal} evolutionary transitions in individuality, in which the higher-level replicating entity is derived from the combination of cooperating kin that have entwined their long-term fates \citep{west2015major}.
Multicellular organisms and eusocial insect colonies exemplify this phenomenon \citep{smith1997major}.
Both are sustained and propagated through the cooperation of lower-level kin.
Although not our focus here, egalitarian transitions --- events in which non-kin unite, such as the genesis of mitochondria by symbiosis of free-living prokaryotes and eukaryotes \citep{smith1997major} --- also constitute essential episodes in natural history.

In nature, major transitions occur rarely and over vast time scales, making them challenging to study.
Recent work in experimental evolution \citep{ratcliff2014experimental, ratcliff2015origins, gulli2019evolution, koschwanez2013improved}, mechanistic modeling \citep{hanschen2015evolutionary, staps2019emergence}, and artificial life \citep{goldsby2012task, goldsby2014evolutionary} complements traditional post hoc approaches focused on characterizing the record of natural history.
These systems each instantiate the evolutionary transition process, allowing targeted manipulations to test hypotheses about the requisites, mechanisms, and evolutionary consequences of fraternal transitions.

Our work here follows closely in the intellectual vein of Goldsby's deme-based artificial life experiments \citep{goldsby2012task, goldsby2014evolutionary}.
In her studies, organisms exist as a group of cells in a segregated, fixed-size subspace.
Organisms compete for a limited number of population slots.
Cells within organisms are controlled by a heritable computer program that can allow them to self-replicate, interact with their environment, and communicate with neighboring cells.

Two modes of cellular reproduction are defined: tissue accretion and offspring generation.
In the former, a cell copies itself into a neighboring position within its subspace.
In the latter, a population slot is cleared to make space for a daughter organism then seeded with a single daughter cell from the parent organism.
Cells grow freely within an organism, but fecundity depends on the collective profile of computational tasks (e.g., logic functions) performed within the organism.
This setup mirrors the dynamics of biological multicellularity, in which cell proliferation may either grow an existing multicellular body or found a new multicell.

Here, rather than enforcing higher-level organisms exist in terms of explicitly segregated subspaces and reproduce through a system-managed mechanism, we demonstrate the emergence of multicellularity in a framework where organisms' spatial distribution and reproductive process is implicit to organisms themselves.
We expect this spatially-unified approach will enable more nuanced interactions between emergent individuals, which may interact in terms of many immediate cell-cell interfaces instead of a single organism-organism interface.
We also expect that cell-cell interfaces between emergent units of individuality may promote the continued evolution of novelty in terms of interactions between competitors, interactions between daughter and parent groups, and, if nested hierarchical individuality were to emerge, fundamentally different building blocks at each level.
The ability to observe and experimentally manipulate fraternal transitions in a digital system where reproductive, developmental, homeostatic, and social  processes occur implicitly within a unified physics will provide unique insights into natural fraternal transitions.

To induce fraternal transitions of individuality within a unified spatial realm, we reward explicitly-registered hereditary groups for cooperation on a distributed resource-collection task.
Here, we extend previous work exploring the selective implications of this scheme by evolving parameters for manually-designed cell-level strategies \citep{moreno2019toward}.
This work employs an event-driven genetic programming representation called  SignalGP which was designed to facilitate dynamic interactions among agents and between agents and their environment \citep{lalejini2018evolving}.
We debut a scheme to scaffold symmetrical agent behavior with respect to each of its grid neighbors.
