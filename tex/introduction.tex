\section{Introduction}

An evolutionary transition of individuality is an event where a single replicating entity breaks apart into independent replicating sub-entities or multiple previously-independent replicating entities consolidate into a unified replicating entity \citep{smith1997major}.
Coming-together transitions in particular are understood as essential to natural history's remarkable record of complexification and diversification \citep{smith1997major} and, within the domain of artificial life, the corresponding but apparently yet-to-be realized concept of open-ended evolution \citep{taylor2016open, banzhaf2016defining}.

Fraternal evolutionary transitions of individuality, our focus here, are a type of coming-together transition in which a new, more complex replicating entity is derived from the combination of cooperating kin which have entwined their long-term fates \citep{west2015major}.
Eusocial insect colonies, orchestrated by differentiated castes of close kin working in concert to sustain and propagate the collective, exemplify this phenomenon \citep{smith1997major}.
Fraternal transitions represent only one type of coming-together transition: egalitarian transitions --- events in which non-kin unite, such as the genesis of Eukaryotes by symbiosis of prokaryotes and bacteria --- also constitute essential episodes in natural history.

Recent work in experimental evolution \citep{ratcliff2014experimental, ratcliff2015origins, gulli2019evolution, koschwanez2013improved}, mechanistic modeling \citep{hanschen2015evolutionary, staps2019emergence}, and artificial life \citep{goldsby2012task, goldsby2014evolutionary} fruitfully complements traditional observational approaches to studying fraternal transitions.
Beyond characterizing the historical record of fraternal transitions with respect to life on Earth, these avenues of inquiry promise to shed special light on the requisites, mechanisms, and evolutionary consequences of fraternal transitions.

Our work here follows closely in the intellectual vein of Goldsby's deme-based artificial life experiments.
In this string of studies, spatially-segregated pockets of cells (``demes'') compete for space in a fixed-size population of demes.
Individual cells are controlled by self-replicating Avida-style computer programs with special instructions that allow them to interact with their environment and with neighboring cells.
Two modes of reproduction are defined under the deme model: within-deme and deme-founding.
In the first, a cell copies itself into a neighboring toroidal tile within its deme.
In the second, a deme slot is cleared in the deme population then seeded with a single cell from the parent deme.
Cells grow freely within demes, but deme fecundity depends on the collective profile of computational tasks (e.g., logic functions) performed within the deme.
This setup mirrors the dynamics of biological multicellularity, in which cell proliferation may either grow an existing multicellular body or spawn a new multicellular body.

However, we pursue a unified spatial model in which higher-order individuality emerges implicitly in terms of the actions of individual cells rather than explicitly in terms of an algorithmic distinction between two spatial scales.
In our view, the merit and significance of such a distinction on purely philosophical grounds is dubious.
However, more concretely, we hope that a unified spatial approach will enable more nuanced interactions between emergent individuals, which may interface in terms of many immediate cell-cell interfaces instead of a single deme-deme interface.
We conjecture that cell-cell interfaces between emergent units of individuality may promote the continued evolution of novelty in terms of interactions between competitors, interactions between propagule and parent groups, and, if nested hierarchical individuality were to emerge, fundamentally different building blocks at each level.

To realize fraternal transitions of individuality within a unified spatial realm, we reward explicitly-registered hereditary groups for cooperation on a distributed resource collection task.
Here, we extend previous work exploring the selective implications of this scheme by evolving parameters for manually-designed cell-level strategies \citep{moreno2019toward}.

Our work employs an event-driven genetic programming representation called SignalGP which was designed to facilitate dynamic interactions among agents and between agents and their environment \citep{lalejini2018evolving}.
We also debut a regulatory extension to the existing SignalGP system and a novel scheme to interface evolving agents with grid-based sensors and actuators which is designed to gird phenotypic cardinal symmetry.
