\section{Introduction}

In a fraternal evolutionary transition of individuality, a new, more complex replicating entity is derived from the combination of cooperating kin which have entwined their long-term fates \citep{west2015major}.
Eusocial insect colonies and multicellular organisms exemplify this phenomenon  \citep{smith1997major}.
In the first case, individual insects (e.g., ants) act in concert towards the subsistence and propagation of the collective (e.g., an ant colony).
In the second, individual cells are joined together , in which only a fraction participate in the germ line.

Fraternal transitions in individuality: a natural example of organisms cooperating to perform tasks and make decisions at the collective level.
it's basically establishing a distributed system!
Like distributed systems, fraternal groups contend with interference and malfunctions.
The groups must be able to continue to function robustly under the death of individual members.
Not only members dying, but also going actively rogue e.g., cancer.

Peto's paradox

In previous work, we have developed a digital system to study fraternal transitions of individuality
blabla
signalgp
bla bla
it's a distributed system

questions:
we are interested in the effects of increasing disruption on major transitions
* will cooperative strategies be selected against under higher mutation rates?
* will we be able to observe a higher apoptosis rate?
* what is the effect of group size


ultimately, evolving multicellular digital organisms with cancer suppression could be interesting from a theoretical or applied perspective --- perhaps ideas could be extracted for distributed systems or cancer treatment.
Showing that our system selects for this kind of behavior (that mutation is bad for groups and worse for larger groups) would be interesting because under the right conditions --- instruction set design, mutational operators, run length, steppingstone functionality, and other considerations we might be able to evolve this functionality!

It is the aim of Artificial Life research to realize engineered systems that exhibit properties of biological life in order to study them, but also with an eye towards applications such as artificial intelligence \citep{bedau2003artificial}.
Studies of evolution have been of particular interest to the community, especially in regard to how organisms are produced with increasing sophistication and complexity \citep{goldsby2017increasing}.
This particular issue is often described as ``open-ended evolution.''
Although precise definitions and measures of open-ended evolution are still being established, this term is generally understood to refer to evolving systems that exhibit the continued production of novelty \citep{taylor2016open}.
Evolutionary transitions in individuality, which are key to the complexification and diversification of biological life \citep{smith1997major}, have been highlighted as key research targets with respect to the question of open-ended evolution \citep{ray1996evolving, banzhaf2016defining}.

In order to study evolutionary transitions of individuality, we must devise a system in which we expect such transitions to occur in a detectable manner.
To this end, we introduce the DISHTINY (DIStributed Hierarchical Transitions of IndividualitY) platform, which seeks to achieve this goal by explicitly registering organisms in cooperating groups that coordinate spatiotemporally to maximize the harvest of a resource.
Detection of a transition of individuality in DISHTINY is accomplished by identifying resource-sharing and reproductive division of labor among organisms registered to the same cooperating group.
Our system is designed such that hierarchal transitions across an arbitrary number of levels of individuality can be selected for and detected.
We have focused this system on a rigid form of major transition using simple organisms, but the underlying principles that we are studying here can be applied to a wide range of artificial life systems.
Furthermore, DISHTINY is decentralized;
it is amenable to massive parallelization via distributed computing.
We believe that such scalability --- with respect to both concept and implementation --- is an essential consideration in the pursuit of artificial systems capable of generating complexity and novelty rivaling that of biological life via open-ended evolution.

%@CAO, I'm cutting these for now to save space.
%We begin by laying out the design of the DISHTINY platform.
%Then, we introduce a model organism used to test for selection for high-level individuality in DISHTINY.
%Finally, we discuss the results of ecological competition and evolutionary experiments, which demonstrate selection for --- and emergence of --- high-level individuality in DISHTINY.
