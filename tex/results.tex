\section{Results and Discussion}

We begin by broadly assessing the selective consequences of each treatment with respect to the functional and reproductive cooperation characteristic of transitions in evolutionary individuality.
Section \label{sec:reproductive-cooperation} analyzes the relationship between same-channel signaling network context and resource sharing.
Section \label{sec:resource-sharing} analyzes the relationship between same-channel signaling network context and cell reproduction behavior.
Then, section \ref{sec:life-histories} surveys observed multicellular life histories.

Subsequent sections describe the mechanisms and fitness effects of notable phenotypes that evolved in individual replicates.
Section \ref{sec:gene-regulation} reports a group-replication strategy mediated by gene regulation where endogenous propagule groups arise, growing to destroy the parent group.
Section \ref{sec:gradient-conditioned-behavior} describes a dynamic strategy where cells condition their own resource-sharing behavior based on a resource gradient.
Section \ref{sec:morphology} examines an example of morphological patterning of same-channel signaling networks.
Section \ref{sec:cell-cell-messaging} presents two examples of adaptive cell-cell messaging.
In the first, cell-cell messaging disrupts directional and spatial uniformity of resource sharing.
In the second, cell-cell messaging appears to intensify expression of a contextual tit-for-tat policy between same-channel signaling network groups.
Finally, section \ref{sec:apoptosis} examines replicates where widespread apoptosis evolved.

\subsection{Case Studies: Life Histories} \label{sec:life-histories}

Although cooperative cell-level phenotypes were common among evolved same-channel signaling networks, across replicates functional and reproductive cooperation arose by a diverse set of qualitative life histories.
\begin{figure*}[!htbp]
\begin{center}

\begin{subfigure}[b]{\textwidth}
\adjincludegraphics[width=0.18\textwidth, trim={{.66\width} {.66\width} {.0\width} {.0\width}}, clip]{lifecycle/transfer-paint/seed=1004+title=directional_propagule_viz+treat=resource-wave__channelsense-yes__nlev-two+update=1048896+_data_hathash_hash=06a65518d5588ef3+_script_fullcat_hash=8b1f57a580a67198+_source_hash=ffbe01c-clean+ext=}
\adjincludegraphics[width=0.18\textwidth, trim={{.66\width} {.66\width} {.0\width} {.0\width}}, clip]{lifecycle/transfer-paint/seed=1004+title=directional_propagule_viz+treat=resource-wave__channelsense-yes__nlev-two+update=1048928+_data_hathash_hash=06a65518d5588ef3+_script_fullcat_hash=8b1f57a580a67198+_source_hash=ffbe01c-clean+ext=}
\adjincludegraphics[width=0.18\textwidth, trim={{.66\width} {.66\width} {.0\width} {.0\width}}, clip]{lifecycle/transfer-paint/seed=1004+title=directional_propagule_viz+treat=resource-wave__channelsense-yes__nlev-two+update=1048960+_data_hathash_hash=06a65518d5588ef3+_script_fullcat_hash=8b1f57a580a67198+_source_hash=ffbe01c-clean+ext=}
\adjincludegraphics[width=0.18\textwidth, trim={{.66\width} {.66\width} {.0\width} {.0\width}}, clip]{lifecycle/transfer-paint/seed=1004+title=directional_propagule_viz+treat=resource-wave__channelsense-yes__nlev-two+update=1049024+_data_hathash_hash=06a65518d5588ef3+_script_fullcat_hash=8b1f57a580a67198+_source_hash=ffbe01c-clean+ext=}
\adjincludegraphics[width=0.18\textwidth, trim={{.66\width} {.66\width} {.0\width} {.0\width}}, clip]{lifecycle/transfer-paint/seed=1004+title=directional_propagule_viz+treat=resource-wave__channelsense-yes__nlev-two+update=1049408+_data_hathash_hash=06a65518d5588ef3+_script_fullcat_hash=8b1f57a580a67198+_source_hash=ffbe01c-clean+ext=}
\caption{Spawn TODO}
\label{fig:TODO}
\end{subfigure}

\begin{subfigure}[b]{\textwidth}
\adjincludegraphics[width=0.18\textwidth, trim={{.0\width} {.0\width} {.66\width} {.66\width}}, clip]{lifecycle/replace-paint/seed=1023+title=directional_propagule_viz+treat=resource-wave__channelsense-yes__nlev-two+update=1048576+_data_hathash_hash=39aa6b64134daefa+_script_fullcat_hash=8b1f57a580a67198+_source_hash=ffbe01c-clean+ext=}
\adjincludegraphics[width=0.18\textwidth, trim={{.0\width} {.0\width} {.66\width} {.66\width}}, clip]{lifecycle/replace-paint/seed=1023+title=directional_propagule_viz+treat=resource-wave__channelsense-yes__nlev-two+update=1048648+_data_hathash_hash=39aa6b64134daefa+_script_fullcat_hash=8b1f57a580a67198+_source_hash=ffbe01c-clean+ext=}
\adjincludegraphics[width=0.18\textwidth, trim={{.0\width} {.0\width} {.66\width} {.66\width}}, clip]{lifecycle/replace-paint/seed=1023+title=directional_propagule_viz+treat=resource-wave__channelsense-yes__nlev-two+update=1048720+_data_hathash_hash=39aa6b64134daefa+_script_fullcat_hash=8b1f57a580a67198+_source_hash=ffbe01c-clean+ext=}
\adjincludegraphics[width=0.18\textwidth, trim={{.0\width} {.0\width} {.66\width} {.66\width}}, clip]{lifecycle/replace-paint/seed=1023+title=directional_propagule_viz+treat=resource-wave__channelsense-yes__nlev-two+update=1048792+_data_hathash_hash=39aa6b64134daefa+_script_fullcat_hash=8b1f57a580a67198+_source_hash=ffbe01c-clean+ext=}
\adjincludegraphics[width=0.18\textwidth, trim={{.0\width} {.0\width} {.66\width} {.66\width}}, clip]{lifecycle/replace-paint/seed=1023+title=directional_propagule_viz+treat=resource-wave__channelsense-yes__nlev-two+update=1048864+_data_hathash_hash=39aa6b64134daefa+_script_fullcat_hash=8b1f57a580a67198+_source_hash=ffbe01c-clean+ext=}
\caption{Replace TODO}
\label{fig:TODO}
\end{subfigure}

\begin{subfigure}[b]{\textwidth}
\adjincludegraphics[width=0.18\textwidth, trim={{.0\width} {.66\width} {.66\width} {.0\width}}, clip]{lifecycle/burst-paint/seed=1034+title=directional_propagule_viz+treat=resource-wave__channelsense-yes__nlev-two+update=1048648+_data_hathash_hash=02a94757e9b17a36+_script_fullcat_hash=8b1f57a580a67198+_source_hash=ffbe01c-clean+ext=}
\adjincludegraphics[width=0.18\textwidth, trim={{.0\width} {.66\width} {.66\width} {.0\width}}, clip]{lifecycle/burst-paint/seed=1034+title=directional_propagule_viz+treat=resource-wave__channelsense-yes__nlev-two+update=1048744+_data_hathash_hash=02a94757e9b17a36+_script_fullcat_hash=8b1f57a580a67198+_source_hash=ffbe01c-clean+ext=}
\adjincludegraphics[width=0.18\textwidth, trim={{.0\width} {.66\width} {.66\width} {.0\width}}, clip]{lifecycle/burst-paint/seed=1034+title=directional_propagule_viz+treat=resource-wave__channelsense-yes__nlev-two+update=1048840+_data_hathash_hash=02a94757e9b17a36+_script_fullcat_hash=8b1f57a580a67198+_source_hash=ffbe01c-clean+ext=}
\adjincludegraphics[width=0.18\textwidth, trim={{.0\width} {.66\width} {.66\width} {.0\width}}, clip]{lifecycle/burst-paint/seed=1034+title=directional_propagule_viz+treat=resource-wave__channelsense-yes__nlev-two+update=1048936+_data_hathash_hash=02a94757e9b17a36+_script_fullcat_hash=8b1f57a580a67198+_source_hash=ffbe01c-clean+ext=}
\adjincludegraphics[width=0.18\textwidth, trim={{.0\width} {.66\width} {.66\width} {.0\width}}, clip]{lifecycle/burst-paint/seed=1034+title=directional_propagule_viz+treat=resource-wave__channelsense-yes__nlev-two+update=1049032+_data_hathash_hash=02a94757e9b17a36+_script_fullcat_hash=8b1f57a580a67198+_source_hash=ffbe01c-clean+ext=}
\caption{Burst TODO}
\label{fig:TODO}
\end{subfigure}

\begin{subfigure}[b]{\textwidth}
\adjincludegraphics[width=0.18\textwidth, trim={{.5\width} {.33\width} {.17\width} {.33\width}}, clip]{lifecycle/cell-paint/seed=1026+title=directional_daughter_viz+treat=resource-wave__channelsense-yes__nlev-two+update=1048576+_data_hathash_hash=a22f7463ee6886d7+_script_fullcat_hash=ef865c98cd111636+_source_hash=ffbe01c-clean+ext=}
\adjincludegraphics[width=0.18\textwidth, trim={{.5\width} {.33\width} {.17\width} {.33\width}}, clip]{lifecycle/cell-paint/seed=1026+title=directional_daughter_viz+treat=resource-wave__channelsense-yes__nlev-two+update=1048704+_data_hathash_hash=a22f7463ee6886d7+_script_fullcat_hash=ef865c98cd111636+_source_hash=ffbe01c-clean+ext=}
\adjincludegraphics[width=0.18\textwidth, trim={{.5\width} {.33\width} {.17\width} {.33\width}}, clip]{lifecycle/cell-paint/seed=1026+title=directional_daughter_viz+treat=resource-wave__channelsense-yes__nlev-two+update=1048832+_data_hathash_hash=a22f7463ee6886d7+_script_fullcat_hash=ef865c98cd111636+_source_hash=ffbe01c-clean+ext=}
\adjincludegraphics[width=0.18\textwidth, trim={{.5\width} {.33\width} {.17\width} {.33\width}}, clip]{lifecycle/cell-paint/seed=1026+title=directional_daughter_viz+treat=resource-wave__channelsense-yes__nlev-two+update=1048960+_data_hathash_hash=a22f7463ee6886d7+_script_fullcat_hash=ef865c98cd111636+_source_hash=ffbe01c-clean+ext=}
\adjincludegraphics[width=0.18\textwidth, trim={{.5\width} {.33\width} {.17\width} {.33\width}}, clip]{lifecycle/cell-paint/seed=1026+title=directional_daughter_viz+treat=resource-wave__channelsense-yes__nlev-two+update=1049088+_data_hathash_hash=a22f7463ee6886d7+_script_fullcat_hash=ef865c98cd111636+_source_hash=ffbe01c-clean+ext=}
\caption{Cellular TODO}
\label{fig:TODO}
\end{subfigure}


\caption{
TODO
same-channel signaling networks
}
\label{fig:ko-apoptosis}
\end{center}
\end{figure*}


Figure \ref{fig:lifecycle} compares four life histories evolved under the Nested-Wave treatment.
Example \ref{fig:lifecycle-naive} profiles a naive life history in which --- beyond the cellular progenitor of a propagule group --- the parent and propagule groups exhibit no special cooperative relationship.
In example \ref{fig:lifecycle-adjoin}, propagules repeatedly bud off of parent groups to yield a larger network of persistent parent-child cooperators.
In example \ref{fig:lifecycle-sweep}, propagules are generated at the extremities of parent groups and then rapidly replace most or all of the parent group.
Finally, in example \ref{fig:lifecycle-burst}, propagules are generated at the interior of a parent group and replace it from the inside out.

\subsection{Case Study: Gene Regulation} \label{sec:gene-regulation}

\begin{figure}[!htbp]
\begin{center}

\begin{subfigure}[b]{\linewidth}
\begin{center}

\begin{minipage}[t]{0.28\linewidth}
\centering
\vspace{0pt} % for alignment
\adjincludegraphics[width=\linewidth, trim={{.25\width} {.25\width} {.5\width} {.5\width}}, clip]{knockout/interior_propagule/wildtype/seed=1+title=directional_regulator_viz+treat=resource-wave__channelsense-yes__nlev-two+update=8188+_data_hathash_hash=8b493febd79aad1f+_script_fullcat_hash=90718bb0c6ec4dbd+_source_hash=53a2252-clean+ext=}
\footnotesize Wild type
\end{minipage}
\begin{minipage}[t]{0.28\linewidth}
\centering
\vspace{0pt} % for alignment
\adjincludegraphics[width=\linewidth, trim={{.5\width} {.5\width} {.25\width} {.25\width}}, clip]{knockout/interior_propagule/propaguleknockout/seed=1+title=directional_regulator_viz+treat=resource-wave__channelsense-yes__nlev-two+update=8188+_data_hathash_hash=2b6711db47fb5887+_script_fullcat_hash=90718bb0c6ec4dbd+_source_hash=53a2252-clean+ext=}
\footnotesize Propagule knockout
\end{minipage}
\begin{minipage}[t]{0.28\linewidth}
\centering
\vspace{0pt} % for alignment
\adjincludegraphics[width=\linewidth, trim={{.5\width} {.5\width} {.25\width} {.25\width}}, clip]{knockout/interior_propagule/regulationknockout/seed=1+title=directional_regulator_viz+treat=resource-wave__channelsense-yes__nlev-two+update=8188+_data_hathash_hash=11ab5cdd47ed18c7+_script_fullcat_hash=90718bb0c6ec4dbd+_source_hash=53a2252-clean+ext=}
\footnotesize Regulation knockout
\end{minipage}

\caption{Regulation visualizations}
\label{fig:regulation_visualizations}

\end{center}
\end{subfigure}

\begin{minipage}[t]{\linewidth}
\centering
\vspace{0pt} % for alignment
\begin{subfigure}[b]{\linewidth}
\includegraphics[width=\linewidth]{knockout/interior_propagule/title=interior_propagules+_data_hathash_hash=bb0fa6254f1b7398+_script_fullcat_hash=f738b363bea8c98a+_source_hash=53a2252-clean+ext=}%
\caption{Interior propagule rate by genotype}
\label{fig:interior_propagule_rate}
\end{subfigure}
\end{minipage}%
\hspace*{\fill}


\caption{
Comparison of a wild type strain evolved under the ``Nested-Wave'' treatment exhibiting interior propagule generation with knockouts of gene regulation and explicitly propagule-generating reproduction instructions.
Subfigures \ref{fig:interior_propagule-wt}, \ref{fig:interior_propagule-ko-regulation}, and \ref{fig:interior_propagule-ko-propagule} depict gene regulation at each of a cell's four directional SignalGP instances using a PCA mapping from regulatory state to three-dimensional RGB coordinates, calculated uniquely for each level-one same-channel signaling group.
Black borders divide level-one same-channel signaling groups and white borders divide level-zero same-channel signaling groups.
Figure \ref{fig:interior_propagule_rate} compares the mean number of interior propagules observed per level-one same-channel signaling group.
Error bars indicate 95\% confidence.
View an animation of wild type gene regulation at \url{https://mmore500.com/hopto/t}.
View the wild type strain in a live in-browser simulation at \url{https://mmore500.com/hopto/g}.
}
\label{fig:ko-interior_propagule}
\end{center}
\end{figure}


What mechanism determines the localization and timing of the propagule placement observed in life history example \ref{fig:lifecycle-burst}?
This wild type strain exhibits an irregular, but somewhat concentric, spatial pattern of gene regulation illustrated in Figure \ref{fig:interior_propagule-wt}.
In time-series animation, provided in supplementary material, gene regulation appears to fluctuate dynamically.

To assess mechanistic and adaptive role of gene regulation in this strain, we prepared two knockout strains.
In the first, all gene regulation instructions were replaced with Nop instructions (so that gene regulation values would remain default).
In the second, the reproduction instructions to spawn a propagule were replaced with Nop instructions.
Figures \ref{fig:interior_propagule-ko-regulation} and \ref{fig:interior_propagule-ko-propagule} depict the gene regulation phenotypes of these strains.

Figure \ref{fig:interior_propagule_rate} compares interior propagule generation between the strains, confirming the direct mechanistic role of gene regulation in promoting interior propagule generation (non-overlapping 95\% CI).

In head-to-head match-ups, the wild type strain outcompetes both the regulation-knockout ($20/20$; $p < 0.001$; two-tailed exact test) and the propagule-knockout strains
($20/20$; $p < 0.001$; two-tailed exact test).
The deficiency of the propagule-knockout strain confirms the adaptive role of interior propagule generation.
Likewise, the deficiency of the regulation-knockout strain affirms the adaptive role of gene regulation in the focal wild type strain.
