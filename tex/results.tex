\section{Results and Discussion}

In order to broadly assess the selective consequences of each treatment with respect to the functional and reproductive cooperation characteristic of evolutionary transitions of individuality, we begin with aggregate analysis of the relationship between same-channel signaling network context and resource-sharing and cell reproduction behavior.
Then, in section \ref{sec:life-histories} we will survey observed multicellular life histories.

Subsequent sections describe the mechanistic underpinnings and fitness effects of notable phenotypic traits that evolved in individual replicates.
Section \ref{sec:gene-regulation} reports an endogenous propagule-seeding strategy mediated by gene regulation.
Section \ref{sec:gradient-conditioned-behavior} describes cell behavior plastically conditioned by a resource gradient.
Section \ref{sec:morphology} details a stringy same-channel signaling network morphology.
Section \ref{sec:cell-cell-messaging} provides two examples of adaptive cell-cell messaging.
Finally, section \ref{sec:apoptosis} reviews two replicates where widespread apoptosis evolved.

\subsection{Reproductive Cooperation} \label{sec:reproductive-cooperation}

\begin{figure}[!htbp]
\begin{center}

\includegraphics[width=\columnwidth]{reproduction/title=reproductive_labor_surrounded+_data_hathash_hash=4e1be4c5abfa4b05+_script_fullcat_hash=62ec0af515e8429d+_source_hash=ffbe01c-clean+ext=}

\caption{
TODO
}
\label{fig:reproduction_surrounted}
\end{center}
\end{figure}


Figure \ref{fig:reproduction_surrounded} shows cellular reproduction rates based on context in highest-level same-channel signaling networks.
In the nested and even treatments, this corresponded to the level-one channel and, in the flat treatment where no level-one channel was defined, this corresponded to the level-zero channel.
For all treatments, phenotypes with depressed interior cellular reproduction rates dominated across replicates ($p < 0.05$; $n=40$; bootstrap test).
[TODO just lookin' at non-overlapping 95\% CI's y'all]
All three treatments appear to be sufficient to select for reproductive cooperation among cells.

\subsection{Resource Sharing} \label{sec:resource-sharing}

\begin{figure*}[!htbp]
\begin{center}

\includegraphics[width=\textwidth]{sharing/title=Resource_Transfer_Rate+_data_hathash_hash=e07865a0aee42cf7+_script_fullcat_hash=3612aad527ec4368+_source_hash=ffbe01c-clean+ext=}

\caption{
TODO
}
\label{fig:sharing}
\end{center}
\end{figure*}

\begin{figure*}[!htbp]
\begin{center}

\includegraphics[width=\textwidth]{sharing/title=Resource_Transfer_Rate+_data_hathash_hash=4753625fc7f89aa8+_script_fullcat_hash=86fe60aba91758b5+_source_hash=ffbe01c-clean+ext=}

\caption{
TODO
}
\label{fig:sharing_channelmate}
\end{center}
\end{figure*}


Figure \ref{fig:sharing} overviews evolved resource sharing behavior across cellular contexts.

Replicates in the flat treatment exhibit an especially elevated rate of resource sharing to cell children.
This could perhaps be due to an especial selective pressure to convey resource towards the group periphery.

Also, in the even and flat treatments, more resource was shared to neighbors within a cell's same-channel signaling network's propagule than to an unrelated neighbor ($p < 0.05$; $n=40$; bootstrap test).
Functional cooperation to directly further group reproduction therefore appears to have evolved commonly under these treatments.

Surprisingly, in the nested treatment resource was shared at a higher mean rate among high-level same-channel signaling groups than low-level groups.
This observation is likely due to replicates where level-one same-channel signaling groups were composed of single-cell level-zero same-channel signaling groups (where no or very few opportunities for level-zero resource sharing occurred).

Finally, under all treatments resource was transferred to channelmates at a significantly higher mean rate than to unrelated neighbors ($p < 0.05$; $n=40$; bootstrap test).
[TODO just lookin' at non-overlapping 95\% CI's y'all]
This observation suggests that functional cooperation within same-channel groups might have been a common evolutionary outcome under all three treatments.
However, it could potentially be an artifact of resource sharing between direct cellular kin.

Figure \ref{fig:sharing_channelmate} breaks same-channel resource-sharing apart by cellular kin relation.
In all three treatments, mean sharing to direct-kin channelmates was indeed greater than to other channelmates.
This cold be due to an evolutionary incentive to favor direct cell kin over other channelmates, group-level selection for asymmetric resource flow achieved by preferential sharing, or some combination of the two.
However, in all three treatments mean sharing to non-direct-kin channelmates was also significantly greater than resource sharing to unrelated neighbors ($p < 0.05$; $n=40$; bootstrap test).
[TODO just lookin' at non-overlapping 95\% CI's y'all]
Thus, all three treatments appear to be sufficient to select for functional cooperation among cells.

\subsection{Case Studies: Life Histories} \label{sec:life-histories}

Although functional and reproductive cooperation was ubiquitous among same-channel signaling networks, across replicates these outcomes were realized via diverse set of qualitiative life histories.
\begin{figure*}[!htbp]
\begin{center}

\begin{subfigure}[b]{\textwidth}
\adjincludegraphics[width=0.18\textwidth, trim={{.66\width} {.66\width} {.0\width} {.0\width}}, clip]{lifecycle/transfer-paint/seed=1004+title=directional_propagule_viz+treat=resource-wave__channelsense-yes__nlev-two+update=1048896+_data_hathash_hash=06a65518d5588ef3+_script_fullcat_hash=8b1f57a580a67198+_source_hash=ffbe01c-clean+ext=}
\adjincludegraphics[width=0.18\textwidth, trim={{.66\width} {.66\width} {.0\width} {.0\width}}, clip]{lifecycle/transfer-paint/seed=1004+title=directional_propagule_viz+treat=resource-wave__channelsense-yes__nlev-two+update=1048928+_data_hathash_hash=06a65518d5588ef3+_script_fullcat_hash=8b1f57a580a67198+_source_hash=ffbe01c-clean+ext=}
\adjincludegraphics[width=0.18\textwidth, trim={{.66\width} {.66\width} {.0\width} {.0\width}}, clip]{lifecycle/transfer-paint/seed=1004+title=directional_propagule_viz+treat=resource-wave__channelsense-yes__nlev-two+update=1048960+_data_hathash_hash=06a65518d5588ef3+_script_fullcat_hash=8b1f57a580a67198+_source_hash=ffbe01c-clean+ext=}
\adjincludegraphics[width=0.18\textwidth, trim={{.66\width} {.66\width} {.0\width} {.0\width}}, clip]{lifecycle/transfer-paint/seed=1004+title=directional_propagule_viz+treat=resource-wave__channelsense-yes__nlev-two+update=1049024+_data_hathash_hash=06a65518d5588ef3+_script_fullcat_hash=8b1f57a580a67198+_source_hash=ffbe01c-clean+ext=}
\adjincludegraphics[width=0.18\textwidth, trim={{.66\width} {.66\width} {.0\width} {.0\width}}, clip]{lifecycle/transfer-paint/seed=1004+title=directional_propagule_viz+treat=resource-wave__channelsense-yes__nlev-two+update=1049408+_data_hathash_hash=06a65518d5588ef3+_script_fullcat_hash=8b1f57a580a67198+_source_hash=ffbe01c-clean+ext=}
\caption{Spawn TODO}
\label{fig:TODO}
\end{subfigure}

\begin{subfigure}[b]{\textwidth}
\adjincludegraphics[width=0.18\textwidth, trim={{.0\width} {.0\width} {.66\width} {.66\width}}, clip]{lifecycle/replace-paint/seed=1023+title=directional_propagule_viz+treat=resource-wave__channelsense-yes__nlev-two+update=1048576+_data_hathash_hash=39aa6b64134daefa+_script_fullcat_hash=8b1f57a580a67198+_source_hash=ffbe01c-clean+ext=}
\adjincludegraphics[width=0.18\textwidth, trim={{.0\width} {.0\width} {.66\width} {.66\width}}, clip]{lifecycle/replace-paint/seed=1023+title=directional_propagule_viz+treat=resource-wave__channelsense-yes__nlev-two+update=1048648+_data_hathash_hash=39aa6b64134daefa+_script_fullcat_hash=8b1f57a580a67198+_source_hash=ffbe01c-clean+ext=}
\adjincludegraphics[width=0.18\textwidth, trim={{.0\width} {.0\width} {.66\width} {.66\width}}, clip]{lifecycle/replace-paint/seed=1023+title=directional_propagule_viz+treat=resource-wave__channelsense-yes__nlev-two+update=1048720+_data_hathash_hash=39aa6b64134daefa+_script_fullcat_hash=8b1f57a580a67198+_source_hash=ffbe01c-clean+ext=}
\adjincludegraphics[width=0.18\textwidth, trim={{.0\width} {.0\width} {.66\width} {.66\width}}, clip]{lifecycle/replace-paint/seed=1023+title=directional_propagule_viz+treat=resource-wave__channelsense-yes__nlev-two+update=1048792+_data_hathash_hash=39aa6b64134daefa+_script_fullcat_hash=8b1f57a580a67198+_source_hash=ffbe01c-clean+ext=}
\adjincludegraphics[width=0.18\textwidth, trim={{.0\width} {.0\width} {.66\width} {.66\width}}, clip]{lifecycle/replace-paint/seed=1023+title=directional_propagule_viz+treat=resource-wave__channelsense-yes__nlev-two+update=1048864+_data_hathash_hash=39aa6b64134daefa+_script_fullcat_hash=8b1f57a580a67198+_source_hash=ffbe01c-clean+ext=}
\caption{Replace TODO}
\label{fig:TODO}
\end{subfigure}

\begin{subfigure}[b]{\textwidth}
\adjincludegraphics[width=0.18\textwidth, trim={{.0\width} {.66\width} {.66\width} {.0\width}}, clip]{lifecycle/burst-paint/seed=1034+title=directional_propagule_viz+treat=resource-wave__channelsense-yes__nlev-two+update=1048648+_data_hathash_hash=02a94757e9b17a36+_script_fullcat_hash=8b1f57a580a67198+_source_hash=ffbe01c-clean+ext=}
\adjincludegraphics[width=0.18\textwidth, trim={{.0\width} {.66\width} {.66\width} {.0\width}}, clip]{lifecycle/burst-paint/seed=1034+title=directional_propagule_viz+treat=resource-wave__channelsense-yes__nlev-two+update=1048744+_data_hathash_hash=02a94757e9b17a36+_script_fullcat_hash=8b1f57a580a67198+_source_hash=ffbe01c-clean+ext=}
\adjincludegraphics[width=0.18\textwidth, trim={{.0\width} {.66\width} {.66\width} {.0\width}}, clip]{lifecycle/burst-paint/seed=1034+title=directional_propagule_viz+treat=resource-wave__channelsense-yes__nlev-two+update=1048840+_data_hathash_hash=02a94757e9b17a36+_script_fullcat_hash=8b1f57a580a67198+_source_hash=ffbe01c-clean+ext=}
\adjincludegraphics[width=0.18\textwidth, trim={{.0\width} {.66\width} {.66\width} {.0\width}}, clip]{lifecycle/burst-paint/seed=1034+title=directional_propagule_viz+treat=resource-wave__channelsense-yes__nlev-two+update=1048936+_data_hathash_hash=02a94757e9b17a36+_script_fullcat_hash=8b1f57a580a67198+_source_hash=ffbe01c-clean+ext=}
\adjincludegraphics[width=0.18\textwidth, trim={{.0\width} {.66\width} {.66\width} {.0\width}}, clip]{lifecycle/burst-paint/seed=1034+title=directional_propagule_viz+treat=resource-wave__channelsense-yes__nlev-two+update=1049032+_data_hathash_hash=02a94757e9b17a36+_script_fullcat_hash=8b1f57a580a67198+_source_hash=ffbe01c-clean+ext=}
\caption{Burst TODO}
\label{fig:TODO}
\end{subfigure}

\begin{subfigure}[b]{\textwidth}
\adjincludegraphics[width=0.18\textwidth, trim={{.5\width} {.33\width} {.17\width} {.33\width}}, clip]{lifecycle/cell-paint/seed=1026+title=directional_daughter_viz+treat=resource-wave__channelsense-yes__nlev-two+update=1048576+_data_hathash_hash=a22f7463ee6886d7+_script_fullcat_hash=ef865c98cd111636+_source_hash=ffbe01c-clean+ext=}
\adjincludegraphics[width=0.18\textwidth, trim={{.5\width} {.33\width} {.17\width} {.33\width}}, clip]{lifecycle/cell-paint/seed=1026+title=directional_daughter_viz+treat=resource-wave__channelsense-yes__nlev-two+update=1048704+_data_hathash_hash=a22f7463ee6886d7+_script_fullcat_hash=ef865c98cd111636+_source_hash=ffbe01c-clean+ext=}
\adjincludegraphics[width=0.18\textwidth, trim={{.5\width} {.33\width} {.17\width} {.33\width}}, clip]{lifecycle/cell-paint/seed=1026+title=directional_daughter_viz+treat=resource-wave__channelsense-yes__nlev-two+update=1048832+_data_hathash_hash=a22f7463ee6886d7+_script_fullcat_hash=ef865c98cd111636+_source_hash=ffbe01c-clean+ext=}
\adjincludegraphics[width=0.18\textwidth, trim={{.5\width} {.33\width} {.17\width} {.33\width}}, clip]{lifecycle/cell-paint/seed=1026+title=directional_daughter_viz+treat=resource-wave__channelsense-yes__nlev-two+update=1048960+_data_hathash_hash=a22f7463ee6886d7+_script_fullcat_hash=ef865c98cd111636+_source_hash=ffbe01c-clean+ext=}
\adjincludegraphics[width=0.18\textwidth, trim={{.5\width} {.33\width} {.17\width} {.33\width}}, clip]{lifecycle/cell-paint/seed=1026+title=directional_daughter_viz+treat=resource-wave__channelsense-yes__nlev-two+update=1049088+_data_hathash_hash=a22f7463ee6886d7+_script_fullcat_hash=ef865c98cd111636+_source_hash=ffbe01c-clean+ext=}
\caption{Cellular TODO}
\label{fig:TODO}
\end{subfigure}


\caption{
TODO
same-channel signaling networks
}
\label{fig:ko-apoptosis}
\end{center}
\end{figure*}


Figure \ref{fig:lifecycle} compares four life histories evolved under the nested treatment.
In example \ref{fig:lifecycle-coalesce}, propagules repeatedly bud off of parent groups to yield a larger network of persistent parent-child cooperators.
In example, \ref{fig:lifecycle-sweep}, propagules are generated at the extremities of parent groups and then rapidly replace most or all of the parent group.
In example, \ref{fig:lifecycle-burst}, propagules are generated at the interior of a parent group and replace it from the inside out.
Finally, example \ref{fig:lifecycle-naive} profiles a more naive life history in which --- beyond the cellular progenitor of a propagule group --- the parent and propagule groups exhibit no special cooperative relationship.

\subsection{Case Study: Gene Regulation} \label{sec:gene-regulation}

\begin{figure}[!htbp]
\begin{center}

\begin{subfigure}[b]{\linewidth}
\begin{center}

\begin{minipage}[t]{0.28\linewidth}
\centering
\vspace{0pt} % for alignment
\adjincludegraphics[width=\linewidth, trim={{.25\width} {.25\width} {.5\width} {.5\width}}, clip]{knockout/interior_propagule/wildtype/seed=1+title=directional_regulator_viz+treat=resource-wave__channelsense-yes__nlev-two+update=8188+_data_hathash_hash=8b493febd79aad1f+_script_fullcat_hash=90718bb0c6ec4dbd+_source_hash=53a2252-clean+ext=}
\footnotesize Wild type
\end{minipage}
\begin{minipage}[t]{0.28\linewidth}
\centering
\vspace{0pt} % for alignment
\adjincludegraphics[width=\linewidth, trim={{.5\width} {.5\width} {.25\width} {.25\width}}, clip]{knockout/interior_propagule/propaguleknockout/seed=1+title=directional_regulator_viz+treat=resource-wave__channelsense-yes__nlev-two+update=8188+_data_hathash_hash=2b6711db47fb5887+_script_fullcat_hash=90718bb0c6ec4dbd+_source_hash=53a2252-clean+ext=}
\footnotesize Propagule knockout
\end{minipage}
\begin{minipage}[t]{0.28\linewidth}
\centering
\vspace{0pt} % for alignment
\adjincludegraphics[width=\linewidth, trim={{.5\width} {.5\width} {.25\width} {.25\width}}, clip]{knockout/interior_propagule/regulationknockout/seed=1+title=directional_regulator_viz+treat=resource-wave__channelsense-yes__nlev-two+update=8188+_data_hathash_hash=11ab5cdd47ed18c7+_script_fullcat_hash=90718bb0c6ec4dbd+_source_hash=53a2252-clean+ext=}
\footnotesize Regulation knockout
\end{minipage}

\caption{Regulation visualizations}
\label{fig:regulation_visualizations}

\end{center}
\end{subfigure}

\begin{minipage}[t]{\linewidth}
\centering
\vspace{0pt} % for alignment
\begin{subfigure}[b]{\linewidth}
\includegraphics[width=\linewidth]{knockout/interior_propagule/title=interior_propagules+_data_hathash_hash=bb0fa6254f1b7398+_script_fullcat_hash=f738b363bea8c98a+_source_hash=53a2252-clean+ext=}%
\caption{Interior propagule rate by genotype}
\label{fig:interior_propagule_rate}
\end{subfigure}
\end{minipage}%
\hspace*{\fill}


\caption{
Comparison of a wild type strain evolved under the ``Nested-Wave'' treatment exhibiting interior propagule generation with knockouts of gene regulation and explicitly propagule-generating reproduction instructions.
Subfigures \ref{fig:interior_propagule-wt}, \ref{fig:interior_propagule-ko-regulation}, and \ref{fig:interior_propagule-ko-propagule} depict gene regulation at each of a cell's four directional SignalGP instances using a PCA mapping from regulatory state to three-dimensional RGB coordinates, calculated uniquely for each level-one same-channel signaling group.
Black borders divide level-one same-channel signaling groups and white borders divide level-zero same-channel signaling groups.
Figure \ref{fig:interior_propagule_rate} compares the mean number of interior propagules observed per level-one same-channel signaling group.
Error bars indicate 95\% confidence.
View an animation of wild type gene regulation at \url{https://mmore500.com/hopto/t}.
View the wild type strain in a live in-browser simulation at \url{https://mmore500.com/hopto/g}.
}
\label{fig:ko-interior_propagule}
\end{center}
\end{figure}


What mechanism determines the localization and timing of the propagule placement obseved in life history example \ref{fig:lifecycle-burst}?
This wild type strain exhibits an irregular, but somewhat concentric, spatial pattern of gene regulation illustrated in Figure \ref{fig:interior_propagule-wt}.
In time-series animation, provided in supplementary material, gene regulation appears to fluctuate dynamically.

To assess mechanistic and adaptive role of gene regulation in this strain, we prepared two knockout strains.
In the first, all gene regulation instructions were replaced with \textbf{Nop} instructions (so that gene regulation values would remain default).
In the second, the reproduction instructions to spawn a propagule were replaced with \textbf{Nop} instructions.
Figures \ref{fig:interior_propagule-ko-regulation} and \ref{fig:interior_propagule-ko-propagule} depict the gene regulation phenotypes of these strains.

Figure \ref{fig:interior_propagule_rate} compares interior propagule generation between the strains, confirming the direct mechanistic role of gene regulation in promoting interior propagule generation ($p < 0.05$; bootstrap test).
[TODO just lookin' at non-overlapping 95\% CI's y'all]

In head-to-head match-ups, the wild type strain outcompetes both the regulation-knockout ($20/20$; $p < 0.001$; two-tailed exact test) and the propagule-knockout strains
($20/20$; $p < 0.001$; two-tailed exact test).
The deficiency of the propagule-knockout strain confirms the adaptive role of interior propagule generation.
Likewise, the deficiency of the regulation-knockout strain affirms the adaptive role of gene regulation in the focal wild type strain.

\subsection{Case Study: Gradient-conditioned Cell Behavior} \label{sec:gradient-conditioned-behavior}

\begin{figure}[!htbp]
\begin{center}

\hspace*{\fill}%
\begin{minipage}[t]{0.45\columnwidth}
\centering
\vspace{0pt} % for alignment
\begin{subfigure}[b]{\textwidth}
\adjincludegraphics[width=\textwidth, trim={{.0\width} {.0\width} {.5\width} {.5\width}}, clip]{knockout/stockpiletrigger-sharing/wildtype/seed=1+title=directional_sharing_viz+treat=resource-wave__channelsense-yes__nlev-two+update=7172+_data_hathash_hash=d856da4ae5863122+_script_fullcat_hash=3a1e851383e0ffd4+_source_hash=53a2252-clean+ext=}%
\caption{Wild type}
\label{fig:TODO}
\end{subfigure}
\end{minipage}%
\hfill
\begin{minipage}[t]{0.45\columnwidth}
\centering
\vspace{0pt} % for alignment
\begin{subfigure}[b]{\textwidth}
\adjincludegraphics[width=\textwidth, trim={{.0\width} {.0\width} {.5\width} {.5\width}}, clip]{knockout/stockpiletrigger-sharing/knockout/seed=1+title=directional_sharing_viz+treat=resource-wave__channelsense-yes__nlev-two+update=7172+_data_hathash_hash=6ab6ade50c5344bc+_script_fullcat_hash=3a1e851383e0ffd4+_source_hash=53a2252-clean+ext=}%
\caption{Stockpile trigger knockout}
\label{fig:TODO}
\end{subfigure}
\end{minipage}%
\hspace*{\fill}

\caption{
Comparison of a wild type strain and corresponding resource-sensing knockout strain.
Color coding represents the amount of incoming shared resource.
White represents no incoming messages and the magenta to blue gradient runs from one incoming message to the maximum observed incoming message traffic.
}
\label{fig:ko-stockpiletrigger-sharing}
\end{center}
\end{figure}


We discovered a strain using resource concentration to regulate directionality of resource sharing in a manner somewhat akin to morphogenic patterning.
This strain's wild type outcompeted a variant with knock out of capacity to asses relative richness of neighboring resource stockpiles ($20/20$; $p < 0.001$; two-tailed exact test).
Figure \ref{fig:ko-stockpiletrigger-sharing} contrasts the wild type resource-sharing phenotype  with the more sparse knockout resource-sharing phenotype.

\subsection{Case Study: Morphology} \label{sec:morphology}

\begin{figure}[!htbp]
\begin{center}

\hspace*{\fill}%
\begin{minipage}[t]{0.45\columnwidth}
\centering
\vspace{0pt} % for alignment
\begin{subfigure}[b]{\textwidth}
\adjincludegraphics[width=\textwidth, trim={{.0\width} {.0\width} {.5\width} {.5\width}}, clip]{knockout/morphology/wildtype/seed=1+title=channel_viz+treat=resource-even__channelsense-yes__nlev-two+update=8188+_data_hathash_hash=cb64cdf045bc6049+_script_fullcat_hash=7e789c981e3d0e4f+_source_hash=53a2252-clean+ext=}
\caption{Wild type}
\label{fig:TODO}
\end{subfigure}
\end{minipage}%
\hfill
\begin{minipage}[t]{0.45\columnwidth}
\centering
\vspace{0pt} % for alignment
\begin{subfigure}[b]{\textwidth}
\adjincludegraphics[width=\textwidth, trim={{.0\width} {.0\width} {.5\width} {.5\width}}, clip]{knockout/morphology/knockout/seed=1+title=channel_viz+treat=resource-even__channelsense-yes__nlev-two+update=8188+_data_hathash_hash=9a4119947348e91d+_script_fullcat_hash=7e789c981e3d0e4f+_source_hash=53a2252-clean+ext=}%
\caption{Intracell messaging knockout}
\label{fig:TODO}
\end{subfigure}
\end{minipage}%
\hspace*{\fill}


\begin{subfigure}[b]{\columnwidth}
\end{subfigure}

\begin{subfigure}[b]{\columnwidth}
\end{subfigure}%

\caption{
TODO
same-channel signaling networks
}
\label{fig:ko-morphology}
\end{center}
\end{figure}


One of the more striking examples of genetically-encoded same-channel signaling network patterning, in which level-zero same-channel signaling groups arranged as elongated strings, arose in a even treatment replicate.
Figure \ref{fig:morphology-wt} provides a snapshot of this strain's same-channel signaling morphology.
Knocking out intracell messaging disrupts the stringy arrangement of same-channel signaling groups, shown in Figure \ref{fig:morphology-ko}.
Figure \ref{fig:morphology-shape} confirms the genetic basis of the trait.
Cells from the wild-type strain more often neighbor only zero or one level-zero same-channel cells ($p < 0.05$; bootstrap test).
[TODO just lookin' at non-overlapping 95\% CI's y'all]
In contrast, knockout strain cells more frequently neighbor three or four level-zero same-channel cells ($p < 0.05$; bootstrap test).
[TODO just lookin' at non-overlapping 95\% CI's y'all]

However, competition experiments between the wild type and knockout strain failed to establish a fitness differential ($6/20$).
Thus, it seems this trait emerged either by drift, as the genetic background of a selective sweep, or --- perhaps less likely --- was advantageous against a divergent competitor earlier in evolutionary history.

\subsection{Case Studies: Cell-cell Messaging} \label{sec:cell-cell-messaging}

\begin{figure*}[!htbp]
\begin{center}

\begin{minipage}[t]{\columnwidth}
\hspace*{\fill}%
\begin{minipage}[t]{0.45\columnwidth}
\centering
\vspace{0pt} % for alignment
\begin{subfigure}[b]{\textwidth}
\adjincludegraphics[width=\textwidth, trim={{.0\width} {.0\width} {.5\width} {.5\width}}, clip]{knockout/intermessaging-sharing/wildtype/seed=1+title=directional_messaging_viz+treat=resource-wave__channelsense-yes__nlev-onebig+update=7172+_data_hathash_hash=f9e2a8ff33bf7745+_script_fullcat_hash=6b7e0389992dd616+_source_hash=53a2252-clean+ext=}%
\caption{Wild type intercell messaging}
\label{fig:TODO}
\end{subfigure}
\end{minipage}%
\hfill
\begin{minipage}[t]{0.45\columnwidth}
\centering
\vspace{0pt} % for alignment
\begin{subfigure}[b]{\textwidth}
\adjincludegraphics[width=\textwidth, trim={{.0\width} {.0\width} {.5\width} {.5\width}}, clip]{knockout/intermessaging-sharing/knockout/seed=1+title=directional_messaging_viz+treat=resource-wave__channelsense-yes__nlev-onebig+update=7172+_data_hathash_hash=ffdeb1c77dd012e1+_script_fullcat_hash=6b7e0389992dd616+_source_hash=53a2252-clean+ext=}%
\caption{Intercell messaging knockout intercell messaging}
\label{fig:TODO}
\end{subfigure}
\end{minipage}%
\hspace*{\fill}

\hspace*{\fill}%
\begin{minipage}[t]{0.45\columnwidth}
\centering
\vspace{0pt} % for alignment
\begin{subfigure}[b]{\textwidth}
\adjincludegraphics[width=\textwidth, trim={{.0\width} {.0\width} {.5\width} {.5\width}}, clip]{knockout/intermessaging-sharing/wildtype/seed=1+title=directional_sharing_viz+treat=resource-wave__channelsense-yes__nlev-onebig+update=7172+_data_hathash_hash=f9e2a8ff33bf7745+_script_fullcat_hash=3a1e851383e0ffd4+_source_hash=53a2252-clean+ext=}%
\caption{Wild type resource sharing}
\label{fig:TODO}
\end{subfigure}
\end{minipage}%
\hfill
\begin{minipage}[t]{0.45\columnwidth}
\centering
\vspace{0pt} % for alignment
\begin{subfigure}[b]{\textwidth}
\adjincludegraphics[width=\textwidth, trim={{.0\width} {.0\width} {.5\width} {.5\width}}, clip]{knockout/intermessaging-sharing/knockout/seed=1+title=directional_sharing_viz+treat=resource-wave__channelsense-yes__nlev-onebig+update=7172+_data_hathash_hash=ffdeb1c77dd012e1+_script_fullcat_hash=3a1e851383e0ffd4+_source_hash=53a2252-clean+ext=}%
\caption{Intercell messaging knockout resource sharing}
\label{fig:TODO}
\end{subfigure}
\end{minipage}%
\hspace*{\fill}

\hspace*{\fill}%
\begin{minipage}[t]{0.45\columnwidth}
\centering
\vspace{0pt} % for alignment
\begin{subfigure}[b]{\textwidth}
\adjincludegraphics[width=\textwidth, trim={{.0\width} {.0\width} {.5\width} {.5\width}}, clip]{knockout/intermessaging-sharing/wildtype/seed=1+title=stockpile_viz+treat=resource-wave__channelsense-yes__nlev-onebig+update=7172+_data_hathash_hash=f9e2a8ff33bf7745+_script_fullcat_hash=4c8152cbf92e0da6+_source_hash=53a2252-clean+ext=}%
\caption{Wild type intercell resource stockpiles}
\label{fig:TODO}
\end{subfigure}
\end{minipage}%
\hfill
\begin{minipage}[t]{0.45\columnwidth}
\centering
\vspace{0pt} % for alignment
\begin{subfigure}[b]{\textwidth}
\adjincludegraphics[width=\textwidth, trim={{.0\width} {.0\width} {.5\width} {.5\width}}, clip]{knockout/intermessaging-sharing/knockout/seed=1+title=stockpile_viz+treat=resource-wave__channelsense-yes__nlev-onebig+update=7172+_data_hathash_hash=ffdeb1c77dd012e1+_script_fullcat_hash=4c8152cbf92e0da6+_source_hash=53a2252-clean+ext=}%
\caption{Intercell messaging knockout resource stockpiles}
\label{fig:TODO}
\end{subfigure}
\end{minipage}%
\hspace*{\fill}
\end{minipage}%
\begin{minipage}[t]{\columnwidth}
\hspace*{\fill}%
\begin{minipage}[t]{\textwidth}
\centering
\vspace{0pt} % for alignment
\begin{subfigure}[b]{\textwidth}
\includegraphics[width=\textwidth]{knockout/intermessaging-sharing/title=sharingdirection+_data_hathash_hash=59f6520a17fb3ad8+_script_fullcat_hash=97aad8dce5e50084+_source_hash=53a2252-clean+ext=}%
\caption{Wild type versus knockout net sharing variance in directionality}
\label{fig:TODO}
\end{subfigure}
\end{minipage}%
\hfill
\begin{minipage}[t]{\textwidth}
\centering
\vspace{0pt} % for alignment
\begin{subfigure}[b]{\textwidth}
\includegraphics[width=\textwidth]{knockout/intermessaging-sharing/title=sharingquadrant+_data_hathash_hash=586f3c805332c323+_script_fullcat_hash=6e8aa37a96d9d7a9+_source_hash=53a2252-clean+ext=}%
\caption{Wild type versus knockout net sharing variance in localization}
\label{fig:TODO}
\end{subfigure}
\end{minipage}%
\hfill
\begin{minipage}[t]{\textwidth}
\centering
\vspace{0pt} % for alignment
\begin{subfigure}[b]{\textwidth}
\includegraphics[width=\textwidth]{knockout/intermessaging-sharing/title=fractionresevoir+_data_hathash_hash=7ce9af7e8fe0699b+_script_fullcat_hash=da31ee3af7ae0208+_source_hash=53a2252-clean+ext=}%
\caption{Wild type versus knockout net sharing variance in localization}
\label{fig:TODO}
\end{subfigure}
\end{minipage}%
\hspace*{\fill}
\end{minipage}

\caption{
TODO
}
\label{fig:ko-apoptosis}
\end{center}
\end{figure*}


We analyzed the phenotypic and fitness implications of cell-cell messaging in two strains: strain A, evolved under the flat treatment, and strain B, evolved under the nested treatment.

Figure \ref{fig:ko-intermessaging-sharing-phen} depicts the cell-cell messaging, resource sharing, and resource stockpile phenotypes of wild type strain A side-by-side with corresponding phenotypes of a cell-cell messaging knockout strain.
In the wild type strain, cell-cell messaging emanates from irregular collection of cells --- in some regions, grid-like and in others more sparse --- broadcasting to all neighboring cells.
Resource sharing appears more widespread in the knockout strain than in the wild type.
However, messaging's effects suppressing resource sharing is neither spatially nor directionally homogeneous.
Relative to the knockout strain, cell-cell messaging increases variance in cardinal directionality of net resource sharing (Figure \ref{fig:ko-intermessaging-sharing-direction}; $p < 0.05$; bootstrap test).
Cell-cell messaging also increases variance of resource sharing density with respect to spatial quadrants demarcated by same-channel signaling group's spatial centroid (Figure \ref{fig:ko-intermessaging-sharing-direction}; $p < 0.001$; bootstrap test).
We used competition experiments to confirm the fitness advantage both of cell-cell messaging ($20/20$; $p < 0.001$; two-tailed exact test) and (using a separate knockout strain) resource sharing ($20/20$; $p < 0.001$; two-tailed exact test).
The fitness advantage of irregularized sharing might stem from a corresponding increase in the fraction of cells with enough resource to reproduce stockpiled ($p < 0.05$; bootstrap test).
[TODO just lookin' at non-overlapping 95\% CI's y'all]


\begin{figure*}[!htbp]
\begin{center}


\begin{minipage}[t]{\columnwidth}
\hspace*{\fill}%
\begin{minipage}[t]{0.05\columnwidth}
\vspace{0pt} % for alignment
\rotatebox{90}{Messaging}%
\end{minipage}%
\hfill
\begin{minipage}[t]{0.45\columnwidth}
\centering
\vspace{0pt} % for alignment
\adjincludegraphics[width=\textwidth, trim={{.0\width} {.49\width} {.66\width} {.17\width}}, clip]{knockout/intermessaging-intergroup_border/wildtype/seed=1+title=directional_messaging_viz+treat=resource-wave__channelsense-yes__nlev-two+update=7168+_data_hathash_hash=3895dfa0dd602b4c+_script_fullcat_hash=6b7e0389992dd616+_source_hash=53a2252-clean+ext=}%
\end{minipage}%
\hfill
\begin{minipage}[t]{0.45\columnwidth}
\centering
\vspace{0pt} % for alignment
\adjincludegraphics[width=\textwidth, trim={{.0\width} {.49\width} {.66\width} {.17\width}}, clip]{knockout/intermessaging-intergroup_border/knockout/seed=1+title=directional_messaging_viz+treat=resource-wave__channelsense-yes__nlev-two+update=7168+_data_hathash_hash=24546cc614406803+_script_fullcat_hash=6b7e0389992dd616+_source_hash=53a2252-clean+ext=}%
\end{minipage}%
\hspace*{\fill}

\hspace*{\fill}%
\begin{minipage}[t]{0.05\columnwidth}
\vspace{0pt} % for alignment
\rotatebox{90}{Parent-Propagule}%
\end{minipage}%
\hfill
\begin{minipage}[t]{0.45\columnwidth}
\centering
\vspace{0pt} % for alignment
\adjincludegraphics[width=\textwidth, trim={{.0\width} {.49\width} {.66\width} {.17\width}}, clip]{knockout/intermessaging-intergroup_border/wildtype/seed=1+title=directional_propagule_viz+treat=resource-wave__channelsense-yes__nlev-two+update=7168+_data_hathash_hash=3895dfa0dd602b4c+_script_fullcat_hash=8b1f57a580a67198+_source_hash=53a2252-clean+ext=}%
\end{minipage}%
\hfill
\begin{minipage}[t]{0.45\columnwidth}
\centering
\vspace{0pt} % for alignment
\adjincludegraphics[width=\textwidth, trim={{.0\width} {.49\width} {.66\width} {.17\width}}, clip]{knockout/intermessaging-intergroup_border/knockout/seed=1+title=directional_propagule_viz+treat=resource-wave__channelsense-yes__nlev-two+update=7168+_data_hathash_hash=24546cc614406803+_script_fullcat_hash=8b1f57a580a67198+_source_hash=53a2252-clean+ext=}%
\end{minipage}%
\hspace*{\fill}

\hspace*{\fill}%
\begin{minipage}[t]{0.05\columnwidth}
\vspace{0pt} % for alignment
\rotatebox{90}{Cell Age}%
\end{minipage}%
\hfill
\begin{minipage}[t]{0.45\columnwidth}
\centering
\vspace{0pt} % for alignment
\adjincludegraphics[width=\textwidth, trim={{.0\width} {.49\width} {.66\width} {.17\width}}, clip]{knockout/intermessaging-intergroup_border/wildtype/seed=1+title=cellage_viz+treat=resource-wave__channelsense-yes__nlev-two+update=7168+_data_hathash_hash=3895dfa0dd602b4c+_script_fullcat_hash=4ac93e074a30cd25+_source_hash=53a2252-clean+ext=}%
\end{minipage}%
\hfill
\begin{minipage}[t]{0.45\columnwidth}
\centering
\vspace{0pt} % for alignment
\adjincludegraphics[width=\textwidth, trim={{.0\width} {.49\width} {.66\width} {.17\width}}, clip]{knockout/intermessaging-intergroup_border/knockout/seed=1+title=cellage_viz+treat=resource-wave__channelsense-yes__nlev-two+update=7168+_data_hathash_hash=24546cc614406803+_script_fullcat_hash=4ac93e074a30cd25+_source_hash=53a2252-clean+ext=}%
\end{minipage}%
\hspace*{\fill}

\hspace*{\fill}%
\begin{minipage}[t]{0.05\columnwidth}
\vspace{0pt} % for alignment
\rotatebox{90}{Resource Stockpile}%
\end{minipage}%
\hfill
\begin{minipage}[t]{0.45\columnwidth}
\centering
\vspace{0pt} % for alignment
\adjincludegraphics[width=\textwidth, trim={{.0\width} {.49\width} {.66\width} {.17\width}}, clip]{knockout/intermessaging-intergroup_border/wildtype/seed=1+title=stockpile_viz+treat=resource-wave__channelsense-yes__nlev-two+update=7168+_data_hathash_hash=3895dfa0dd602b4c+_script_fullcat_hash=4c8152cbf92e0da6+_source_hash=53a2252-clean+ext=}%
\end{minipage}%
\hfill
\begin{minipage}[t]{0.45\columnwidth}
\centering
\vspace{0pt} % for alignment
\adjincludegraphics[width=\textwidth, trim={{.0\width} {.49\width} {.66\width} {.17\width}}, clip]{knockout/intermessaging-intergroup_border/knockout/seed=1+title=stockpile_viz+treat=resource-wave__channelsense-yes__nlev-two+update=7168+_data_hathash_hash=24546cc614406803+_script_fullcat_hash=4c8152cbf92e0da6+_source_hash=53a2252-clean+ext=}%
\end{minipage}%
\hspace*{\fill}

\vspace{1.0ex}

\hspace*{\fill}%
\begin{minipage}[t]{0.05\columnwidth}
\vspace{0pt} % for alignment
\end{minipage}%
\hfill
\begin{minipage}[t]{0.45\columnwidth}
\centering
\vspace{0pt} % for alignment
Wild Type
\end{minipage}%
\hfill
\begin{minipage}[t]{0.45\columnwidth}
\centering
\vspace{0pt} % for alignment
Messaging Knockout
\end{minipage}%
\hspace*{\fill}

\vspace{1.0ex}

\begin{subfigure}{\columnwidth}
  \caption{Phenotype visualizations}
\end{subfigure}

\end{minipage}%
\begin{minipage}[t]{\columnwidth}

\hspace*{\fill}%
\begin{minipage}[t]{\columnwidth}
\centering
\vspace{0pt} % for alignment
\begin{subfigure}[b]{\textwidth}
\includegraphics[width=\textwidth]{knockout/intermessaging-intergroup_border/title=borderturnover+_data_hathash_hash=309c318cf489633e+_script_fullcat_hash=c927ae5b721fac54+_source_hash=53a2252-clean+ext=}%
\caption{Border turnover rate}
\label{fig:TODO}
\end{subfigure}
\end{minipage}%
\hspace*{\fill}

\vspace{1ex}

\hspace*{\fill}%
\begin{minipage}[t]{\columnwidth}
\centering
\vspace{0pt} % for alignment
\begin{subfigure}[b]{\textwidth}
\includegraphics[width=\textwidth]{knockout/intermessaging-intergroup_border/title=propcold+_data_hathash_hash=309c318cf489633e+_script_fullcat_hash=c927ae5b721fac54+_source_hash=53a2252-clean+ext=}%
\caption{Border stability}
\label{fig:TODO}
\end{subfigure}
\end{minipage}%
\hspace*{\fill}

\vspace{1ex}

\hspace*{\fill}%
\begin{minipage}[t]{\columnwidth}
\centering
\vspace{0pt} % for alignment
\begin{subfigure}[b]{\textwidth}
\includegraphics[width=\textwidth]{knockout/intermessaging-intergroup_border/title=borderage+_data_hathash_hash=309c318cf489633e+_script_fullcat_hash=c927ae5b721fac54+_source_hash=53a2252-clean+ext=}%
\caption{Border age}
\label{fig:TODO}
\end{subfigure}
\end{minipage}%
\hspace*{\fill}
\end{minipage}

\caption{
TODO
}
\label{fig:ko-apoptosis}
\end{center}
\end{figure*}


Figure \ref{fig:ko-intermessaging-sharing}

Figure \ref{fig:ko-intermessaging-intergroup_border}

\subsection{Case Studies: Apoptosis} \label{sec:apoptosis}

\begin{figure}[!htbp]
\begin{center}

\begin{subfigure}[b]{0.4\columnwidth}
\adjincludegraphics[width=\textwidth, trim={{.5\width} {.5\width} {.0\width} {.0\width}}, clip]{knockout/apoptosis/wildtype/seed=1+title=channel_viz+treat=resource-even__channelsense-yes__nlev-two+update=262144+_data_hathash_hash=9b92a609c3309033+_script_fullcat_hash=7e789c981e3d0e4f+_source_hash=53a2252-clean+ext=}%
\caption{Even treatment wild type}
\label{fig:TODO}
\end{subfigure}
\begin{subfigure}[b]{0.4\columnwidth}
\adjincludegraphics[width=\textwidth, trim={{.5\width} {.5\width} {.0\width} {.0\width}}, clip]{knockout/apoptosis/knockout/seed=1+title=channel_viz+treat=resource-even__channelsense-yes__nlev-two+update=262144+_data_hathash_hash=900abeef45bb9133+_script_fullcat_hash=7e789c981e3d0e4f+_source_hash=53a2252-clean+ext=}%
\caption{Even treatment apoptosis knockout}
\label{fig:TODO}
\end{subfigure}

\begin{subfigure}[b]{0.4\columnwidth}
\adjincludegraphics[width=\textwidth, trim={{.5\width} {.5\width} {.0\width} {.0\width}}, clip]{knockout/apoptosis/wildtype/seed=1+title=channel_viz+treat=resource-wave__channelsense-yes__nlev-onebig+update=8188+_data_hathash_hash=3465df2fce2dc5f4+_script_fullcat_hash=7e789c981e3d0e4f+_source_hash=53a2252-clean+ext=}
\caption{Flat treatment wild type}
\label{fig:TODO}
\end{subfigure}
\begin{subfigure}[b]{0.4\columnwidth}
\adjincludegraphics[width=\textwidth, trim={{.5\width} {.5\width} {.0\width} {.0\width}}, clip]{knockout/apoptosis/knockout/seed=1+title=channel_viz+treat=resource-wave__channelsense-yes__nlev-onebig+update=8188+_data_hathash_hash=9c40470beee1c5b5+_script_fullcat_hash=7e789c981e3d0e4f+_source_hash=53a2252-clean+ext=}%
\caption{Flat treatment apoptosis knockout}
\label{fig:TODO}
\end{subfigure}%

\caption{
TODO
same-channel signaling networks
}
\label{fig:ko-apoptosis}
\end{center}
\end{figure}


Screening replicate evolutionary runs by apoptosis rate flagged two strains with several orders of magnitude greater activity.
In strain A, evolved under the even treatment, apoptosis accounts for 2\% of cell mortality.
In strain B, evolved under the flat treatment, 15\% of mortality is due to apoptosis.

To test the adaptive role of apoptosis in these strains, we performed competition experiments against apoptosis knockout strains, in which all apoptosis instructions were substituted for Nop instructions.
Figure \ref{fig:ko-apoptosis} compares the same-channel wild type phenotypes of these strains to their corresponding knockouts.

Apoptosis contributed significantly to fitness in both strains (strain A: $18/20$, $p < 0.001$, two-tailed exact test; strain B: $20/20$, $p < 0.001$, two-tailed exact test).
The success of strategies incorporating cell suicide is characteristic of a transition from cell-level to collective individuality.
