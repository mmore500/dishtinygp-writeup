\section{Methods}

In order to demonstrate that the DISHTINY platform selects for detectable hierarchical transitions in individuality, we performed experiments where cells controlled by evolving genetic programs evolved open-ended behaviors to make decisions about resource sharing, reproductive timing, and apoptosis.
We will first cover the design of the DISHTINY platform and then describe the cell-like organisms we used to evaluate the platform.

\subsection{Same-Channel Groups}

DISHTINY allows cell-like organisms to replicate across a toroidal grid.
Over discrete time steps (``updates''), the cells can collect a continuous-valued resource.
Once sufficient resource has been accrued, cells may pay $1.0$ resource to place a daughter cell on an adjoining tile of the toroidal grid (i.e., reproduce), replacing any existing cell already there.
Collected resource decays at a rate of 0.1\% per update, incentivizing its quick use.

Cells may benefit from a cooperative resource-collection process conducted by explicitly-registered ``signaling channel'' groupings.
As cells reproduce, they can choose to include offspring in the parent's cooperating ``signaling channel'' group or expel offspring to found a new cooperating ``signaling channel'' group.
These decisions affect the spatial layout of these ``signaling channel'' groups and, in turn, affect individual cells' resource-collection rate.
Medium-size, circular ``signaling channel'' groups tend to collect resource at a greater per-cell rates than large, small, ovular, or irregularly distributed groups.
In order to promote group turnover, we counteract the advantage of established group with a simple aging scheme.
As channel groups age over elapsed updates and elapsed somatic generations, their constituent cells lose the ability regenerate somatic tissue and then, soon after, to collect resource

A complete description of the mechanisms behind these collective resource-collection and group aging mechanisms appears in supplementary materials \ref{sup:resource_collection_process} \ref{sup:channel_group_life_cycle}.

In addition to a potentially functionally cooperative relationship, shared channel IDs --- which may only systematically arise through inheritance --- imply a close hereditary relationship.
In this work, we screen for fraternal transitions in individuality with respect to these same-channel network groups by evaluating resource sharing, antagonistic cell proliferation, and apoptosis across within-group and between-group contexts.

\subsection{Cell-Level Organisms}

Our experiments used cell-level digital organisms controlled by evolving
genetic programs.
We employ the SignalGP representation, which expresses function-like modules of linear program instruction sequences in response to stimuli.
This event-driven paradigm facilitates the evolution of dynamic interactions between digital organisms and their environment (including other digital organisms) \cite{lalejini2018evolving}.

Previous work evolving digital organisms in grid-based problem domains has relied on a single computational instance which designates a direction to act in via an explicit cardinal ``facing'' state or output \cite{goldsby2014evolutionary, goldsby2018serendipitous, grabowski2010early, biswas2014causes, lalejini2018evolving}.
We introduce novel methodology that aims to facilitate the evolution of directionally symmetric phenotypes which effectively incorporate directionally-specific state.
In this work, each cell employs four instances of SignalGP hardware: one ``facing'' each cardinal direction.
These computational instances all execute the same SignalGP program but are otherwise decoupled and may follow independent chains of execution and develop independent regulatory states.
Instances within a cell may coordinate via so-called intracellular messaging.
Supplementary Figure \ref{fig:dishtinygp-cartoon} schematically depicts the configuration of the four SignalGP instances that constitute a single DISHTINY cell as well as the instances of neighboring cells that receive extracellular messages from the focal cell.

\subsection{Treatments}

We screened evolutionary replicates conducted under combinations of two experimental conditions:
\begin{enumerate}
\item flat versus nested hierarchical resource wave/channel-signaling levels and
\item cooperative versus independent resource collection.
\end{enumerate}

The first experimental manipulation explores the effects of hierarchical nesting of kin-sensing and/or functional cooperation.
The second manipulation explores the effects of functional cooperation.

We mix and match these experimental manipulations in three treatments:
\begin{enumerate}
\item one level with even resource (``Flat-Even''; in-browser simulation \url{https://mmore500.com/hopto/i}),
\item one level with wave-based resource (``Flat-Wave''; in-browser simulation \url{https://mmore500.com/hopto/j}),
\item two levels with even resource (``Nested-Even''; in-browser simulation \url{https://mmore500.com/hopto/k}), and
\item two levels with wave-based resource (``Nested-Wave''; in-browser simulation \url{https://mmore500.com/hopto/l}).
\end{enumerate}
