\section{Methods}

In order to demonstrate that the DISHTINY platform selects for detectable hierarchical transitions in individuality, we performed experiments where cell-like organisms evolved parameters to control manually-designed strategies such as resource-sharing, reproductive decision-making, and apoptosis.
We will first cover the design of the DISHTINY platform and then describe the simple cell-like organisms we used to evaluate the platform.

\subsection{DISHTINY}

\input{fig/explanatory.tex}

DISHTINY allows cell-like organisms to replicate across a toroidal grid.
Over discrete timesteps (``updates''), the cells can collect a continuous-valued resource.
Once sufficient resource has been accrued, cells may pay $3.0$ resource to place a daughter cell on an adjoining tile of the toroidal grid (i.e., reproduce), replacing any existing cell already there.
As cells reproduce, they can choose to include offspring in the parent's cooperating ``signaling channel'' group or force offspring to create a new cooperating ``signaling channel'' group.

As shown at the top of Figure \ref{fig:explanatory}, resources appear at a single point then spread outwards update-by-update in a diamond-shaped wave, disappearing when the expanding wave reaches a predefined limit.
Cells must be in a costly ``activated'' state to collect resource as it passes.
The cell at the starting position of a resource wave is automatically activated, and will send the activate signal to neighboring cells on the same signaling channel.
The newly activated cells, in turn, activate their own neighbors registered to the same signaling channel.
Neighbors registered to other signaling channels do not activate.
Each cell, after sending the activation signal, enters a temporary quiescent state so as not to reactivate from the signal.
In this manner, cells sharing a signaling channel activate in concert with the expanding resource wave.
As shown Figure \ref{fig:explanatory}$a,b$, the rate of resource collection for a cell is determined by the size and shape of of its same-channel signaling network;
small or fragmented same-channel signaling networks will frequently miss out on resource as it passes by.

Each cell pays a resource cost when it activates.
This cost is outweighed by the resource collected such that cells that activate in concert with a resource wave derive a net benefit.
Recall, though, that resource waves have a limited extent.
Cells that activate outside the extent of a resource wave or activate out of sync with the resource wave (due to an indirect path from the cell that originated the signal) pay the activation cost but collect no resource.
Cells that frequently activate erroneously use up their resource and die.
In our implementation, organisms that accrue a resource debt of $-5$ or greater are killed.
This erroneous activation scenario is depicted in Figure \ref{fig:explanatory}$c$.

In this manner, ``Goldilocks'' --- not to small and not too big --- signaling networks are selected for.
Based on a randomly chosen starting location, resource wave starting points (seeds) are tiled over the toroidal grid such that the extents of the resource waves touch, but do not overlap.
All waves start and proceed synchronously;
when they complete, the next resource waves are seeded.
This process ensures that selection for ``Goldilocks'' same-channel signaling networks is uniformly distributed over the toroidal grid.

Cells control the size and shape of their same-channel signaling group through strategic reproduction.
Three choices are afforded: whether to reproduce at all, where among the four adjoining tiles of the toroidal grid to place their offspring, and whether the offspring should be registered to the parent's signaling channel or be given a random channel ID (in the range 1 to $2^{64}$).
The probability of channel collision is miniscule: $60 \times 60 \times 50100$ (the grid dimensions times the number of simulation updates) independent channel values will only collide with probability $1 \times 10^{-11}$.
No guarantees are made about the uniqueness of a newly-generated channel ID, but chance collisions are rare.

To ensure turnover of channel groups, a channel generation limit is enforced.
Each time a cell spawns a daughter cell sharing the same channel, the parent's channel generation counter is incremented and the daughter cell's channel generation counter is then initialized to match their  parent's counter.
Once the channel generation counter reaches a limit, in this implementation defined as the resource wave radius, daughter cells may no longer be endowed with the parent's channel ID; a new, randomly-drawn channel ID is assigned to daughter cells instead.

Hierarchical levels are introduced into the system through multiple separate, but overlaid, instantiations of this resource wave/channel-signaling scheme.
We refer to each independent resource wave/channel-signaling system as a ``level.''
In some experimental treatments, we allowed two resource wave/channel-signaling levels, identified here as level one and level two.
On level one, resource waves extended a radius of four toroidal tiles.
On level two they extended a radius of 12 toroidal tiles.
On both levels, activated cells netted $+1.0$ resource from a resource wave, but suffered an activation penalty of $-5.0$ if no resource was available.
Due to the different radii of resource waves on different levels, level one selects for small same-channel signaling networks and level two selects for large same-channel signaling networks.

Cells were marked with two separate channel IDs, one for level one and another for level two.
We enforced hierarchical nesting of same-channel signaling networks during reproduction:
daughter cells may inherit neither channel ID, just the level-two channel ID, or both channel IDs.
Daughter cells may not inherit only the level-one channel ID while having a different level-two channel ID.
The distribution of IDs across the level-two and level-one channels can be envisioned by analogy to political countries and territories.
Each country (i.e., level-two channel network) may have one or many territories (i.e., level-one channel network).
However, no territory spans more than one country.
Figure \ref{fig:outcome_grids} depicts hierarchically nested channel states at the end of three evolutionary runs.

Channel IDs enable straightforward detection of an evolutionary transition in individuality.
Because common channel IDs may only arise systematically through inheritance, common channel IDs indicate a close hereditary relationship in addition to a close cooperative relationship.
Because new channel IDs arise first in a single cell, same-channel signaling networks are reproductively bottlenecked, ensuring meaningful reproductive lineages at the level of the same-channel signaling network.
To recognize an evolutionary transition in individuality, we therefore evaluate
\begin{enumerate}
\item Do cells with the same channel ID choose to share resources (e.g., cooperate)?
\item Is there division of reproductive labor between members of the same channel (e.g., do cells at the interior of a network cede reproduction to those at the periphery?)
\end{enumerate}
If these conditions are met among cells sharing the same level-one channel, a first-level transition in individuality may have occurred.
Likewise, if these conditions are met among cells sharing the same level-two channel, a second-level transition in individuality may have occurred.
In either case, observation of altruistic behavior, such as an apoptosis response to mutation, would further evidence a transition.

\subsection{Organisms}

\begin{figure*}[t]
\begin{center}

\begin{subfigure}[b]{\columnwidth}
\includegraphics[width=\textwidth]{img/signalgp-cartoon}%
\caption{TODO}
\label{fig:TODO}
\end{subfigure}%
\begin{subfigure}[b]{\columnwidth}
\includegraphics[width=\textwidth]{img/dishtinygp-cartoon}
\caption{TODO}
\label{fig:TODO}
\end{subfigure}

\caption{TODO}
\label{fig:signalgp-dishtinygp}

\end{center}
\end{figure*}


We performed our experiments using cell-level digital organisms controlled by evolving SignalGP programs.
SignalGP is genetic programming framework designed around the event-driven programming paradigm \cite{lalejini2018evolving}.
SignalGP programs are collections of independent procedural functions, each equipped with a bit-string tag.
A function is triggered by a signal with affinity that maximally and sufficiently matches its tag.
(A binding threshold of 0.5 was used in these experiments).
Signals may be generated by the environment, received as messages from other agents, or triggered internally by function execution.
Signals, and the ensuing chains of procedural execution they give rise to, are processed pseudo-concurrently by sixteen virtual CPUs.
Fifty virtual CPU cycles are carried out during each update.
Figure \ref{fig:signalgp-dishtinygp}(TODO a) schematically depicts a single SignalGP instance.

Previous work evolving digital organisms to perform grid-based tasks in which a single computational instance designates which direction to act in via an explicit cardinal ``facing'' state or output \cite{goldsby2014evolutionary, goldsby2018serendipitous, grabowski2010early, biswas2014causes, lalejini2018evolving}.
However, in our work each cell employs four instances of SignalGP: one ``facing'' each cardinal direction.
These computational instances all execute the same SignalGP program but are otherwise decoupled and may follow independent chains of execution.
These instances execute single round robin in an order that is randomly drawn at the outset of each update.
The single SignalGP program that is mirrored across the cell's computational instances represents the cell's genome.
Mutation, with standard SignalGP mutation parameters as in \cite{lalejini2018evolving}, is applied to 10\% of daughter cells at birth.

Instances within a cell may send intracellular messages to one another or intercellular messages to a neighboring cell.
Intercellular messages are received by the SignalGP instance that faces the sending cell.
Figure \ref{fig:signalgp-dishtinygp}(TODO b) schematically depicts the configuration of the four SignalGP instances that constitute a single DISHTINY cell as well as the instances of neighboring cells that receive extracellular messages from the focal cell.

Genetic codings that exploit problem-domain symmetries are known to promote evolvability and, ultimately, evolved solution quality \cite{clune2011performance, cheney2014unshackling}.
In grid-based tasks, cardinal symmetry of program action is generally desirable; agents should generally sense and react uniformly across cardinal directions.
In traditional encodings, likely only a small fraction of the genotype space encodes cardinally-symmetric phenotypes.
We suspect that explicitly mirroring computational instances across cardinal directions dramatically increases the fraction of genotype space that encodes cardinally-symmetric phenotypes.
We look forward to exploring the potential evolvability and solution quality implications of this methodological innovation, as well as its potential as an avenue to instantiate ploidy in digital evolution, in further work.

A combination of event-driven environmental triggers and procedural instruction-based sensors detailed in the following two subsections.

\subsection{Instruction Library}

In addition to the generic arithmetic, logic, and program flow instructions in the default SignalGP instruction set, which is detailed in \cite{lalejini2018evolving}, we define the following instructions.
Instructions that involve an extracellular neighbor default to the cell that the executing SignalGP instance is facing, but may be modified by a register-based argument.
Many instructions in the listing are provided in several variants, which are detailed in the accompanying description.

\begin{itemize}
\item \textbf{RNG Draw}
Draw a random value between 0.0 and 1.0 from on board random number generator and store result in register.
\item \textbf{Send/Broadcast Intracellular Message}
Send a message to a single other SignalGP instance within the cell or to all SignalGP instances (except the executing instance) within the current cell.
\item \textbf{Set Stockpile Reserve}
Mark twice the amount of resource as ineligible for sharing.
The amount may be modified by a register-based argument.
\item \textbf{Activate/Deactivate Intercellular Inbox}
Mark or unmark the intercellular inbox in a particular direction to refuse incoming messages.
At cell birth, the inbox is deactivated.
\item \textbf{Share Resource}
Send a proportion of the cell's stockpiled resource to a neighboring cell.
One instruction defaults to sending a large proportion of available resource (50\%) to the neighboring cell.
A second instruction defaults to sending a small proportion of available resource (5\%) to the neighboring cell.
The proportion of available resource can be adjusted by a register-based argument.
\item \textbf{Accept/Decline Sharing}
Mark the cell to decline resource sent by neighbors.
Declined resource is retained by the sending cell (with no resource lost).
Regardless of this instruction, cells with negative resource stockpiles automatically decline shared resource.
At birth, the cell is marked to accept sharing.
\item \textbf{Send/Broadcast Intercellular Message}
Send a message to a single cellular neighbor or to all cellular neighbors.
\item \textbf{Reproduce}
Attempt to spawn a child cell in a particular direction, paid for out of the parent cell's resource stockpile.
If sufficient resource is not available in the cell's stockpile, no resource is action is taken.
Variants of this instruction are defined for each channel ID inheritance level: from endowing the daughter cell with the parental channel IDs across all levels, to endowing the daughter cell with a new level-one channel ID but the parent's level-two channel ID, to endowing the daughter cell with all-new channel IDs.
If a channel generation counter limit has been reached, reproduction is simply attempted at the next highest level; even with channel generation counters maxed out, cells may generate offspring with all-new channel IDs.
\item \textbf{Pause Reproduction}
Pause cellular reproduction in a single direction for the remainder of the current update and for the entire next update.
Variants of this instruction pause reproduction at a certain wave/channel-signaling level or across all channel ID inheritance levels.
\item \textbf{Increase Channel Generation Counter}
Increases the cell's channel generation counter by one.
The amount the cell's generation counter is increased by can be adjusted by register-based argument.
\item \textbf{Apoptosis}
The cell is killed at the end of the current update.
Two variants are defined.
Under the conplete variant, the cell's channel ID partial and complete
\item \textbf{Designate/Revoke Heir} apoptosis 80\% of reproduction cost and own stockpile amount
\item \textbf{Query Own Stockpile}
Sets a designated register to the amount of resource present in the cell's stockpile.
\item \textbf{Query Own Channel Generation Counter} -Lev
This instruction sets a designated register to the value of the cell's channel generation counter.
A variant of this instruction is provided for each wave/channel-signaling level.
\item \textbf{Query Is Neighbor Live}
This instruction sets a designated register to 1 if the neighboring tile contains a live cell and 0 otherwise.
\item \textbf{Query Is Neighbor Channel Set}
This instruction sets a designated register to 1 if the neighboring tile contains active channel IDs and 0 otherwise.
\item \textbf{Query Is Neighbor My Cellular Child}
This instruction sets a designated register to 1 if the neighboring cell is the daughter of the querying cell and 0 otherwise.
\item \textbf{Query Is Neighbor My Cellular Parent}
This instruction sets a designated register to 1 if the neighboring cell is the parent of the querying cell and 0 otherwise.
\item \textbf{Query Does Neighbor's Channel ID Match Mine}
This instruction sets a designated register to 1 if the neighboring cell has the same channel ID as the querying cell and 0 otherwise.
A variant of this instruction is provided for each wave/channel-signaling level.
\item \textbf{Query Does Neighbor's Channel ID Descend From Mine}
This instruction sets a designated register to 1 if the neighboring cell's highest-level channel ID is different from the querying cell's highest-level channel ID, but is descended from the querying cell's channel ID.
This instruction allows a querying cell to sense whether its neighbor is a member of a same-channel group that is a propagule of the querying cell's same-channel group.
\item \textbf{Query Does My Channel ID Descend From Neighbor's}
This instruction sets a designated register to 1 if the querying cell's highest-level channel ID is different from the neighboring cell's highest-level channel ID, but is descended from the neighboring cell's channel ID.
\item \textbf{Query Neighbor's Channel ID}
This instruction sets a designated register to the neighbor's channel ID.
\item \textbf{Query Neighbor's Stockpile}
This instruction sets a designated register to the amount of resource present in the neighbor's stockpile.
\end{itemize}

\subsection{Event Library}

The activating affinity of each event is set at the outset of each experiment using a pseudo random number generator.

\begin{itemize}
\item \textbf{On Update}
This event is triggered at the outset of each simulation update.
\item \textbf{Facing Cellular Child}
This event is triggered at the outset of an update if the SignalGP instance is facing a neighboring cell that is the querying cell's daughter.
\item \textbf{Stockpile Debt}
This event is triggered at the outset of an update if the amount of resource in a cell's stockpile is negative.
\item \textbf{Neighbor's Channel ID Matches Mine} defined per level
This event is triggered at the outset of an update if a SignalGP instance is facing a neighbor cell that shares its channel ID.
A different event is provided for each resource wave/channel-signaling level.
\item \textbf{Neighbor's Channel ID Descends From Mine}
This event is triggered at the outset of an update if the neighboring cell's highest-level channel ID is different from the querying cell's highest-level channel ID, but is descended from the querying cell's channel ID.
This event allows a querying cell to sense whether its neighbor is a member of a same-channel group that is a propagule of the querying cell's same-channel group.
\end{itemize}

\subsection{Treatments}

In this work, we explore three experimental parameters:
\begin{enumerate}
\item dual versus single hierarchical resource wave/channel-signaling levels,
\item wave-based versus uniform resource distribution, and
\begin{itemize}
\item For the uniform resource distribution treatment, irrespective of channel ID each cell receives an identical amount of resource chosen to approximately match the inflow rate of the dual-level wave-based scheme.
\end{itemize}
\item enabled versus disabled channel-sensing events/instructions.
\begin{itemize}
\item For the disabled channel-sensing treatment, all channel-sensing instructions perform no operation and all channel-sensing events are never triggered.
\end{itemize}
\end{enumerate}

The first experimental parameter explores the implications of resource wave size and hierarchical nesting in the DISHTINY model.
The second and third parameters aim to untangle the effects of channel ID with respect to the resource collection task and with respect to kin recognition.

We mix-and match these experimental parameters in six treatments:
\begin{enumerate}
\item two levels, wave-based resource, enabled channel-sensing (``standard''),
\item two levels, even resource, enabled channel-sensing (``even''),
\item two levels, wave-based resource, disabled channel-sensing (``blind''),
\item two levels, even resource, disabled channel-sensing (``even+blind''),
\item one level with wave radius four, wave-based resource, enabled channel-sensing (``small wave''), and
\item one level with wave radius twelve, wave-based resource, enabled channel-sensing (``large wave'').
\end{enumerate}

\subsection{Implementation}

We implemented our experimental system using the Empirical library for scientific software development in C++, available at \url{https://github.com/devosoft/Empirical}.
The code used to perform and analyze our experiments, our figures, data from our experiments, and a live in-browser demo of our system is available via the Open Science Framework at \url{https://osf.io/g58xk/}.
