\section{Methods}

In order to assess selection for group-level individuality, we performed experiments where cells evolved open-ended behaviors to make decisions about resource sharing, reproductive timing, and apoptosis.
% controlled by evolving genetic programs
We will first cover the environmental cells evolved in and then describe the cell-like organisms we used to evaluate the platform.

\subsection{Same-Channel Groups}

In our model, cell-like organisms occupy individual tiles a toroidal grid.
Over discrete time steps (``updates''), cells can collect a resource.
Once sufficient resource has been accrued, cells may pay $1.0$ resource to place a daughter cell on an adjoining tile of the toroidal grid (i.e., reproduce), replacing any existing cell already there.
Collected resource decays at a rate of 0.1\% per update, incentivizing its quick use.

Cells accrue resource via a cooperative resource-collection process conducted by explicitly-registered groupings.
As cells reproduce, they can choose to include offspring in the parent's registered group or expel offspring to found a new cooperating registered group.
These decisions affect the spatial layout of these registered groups and, in turn, affect individual cells' resource-collection rate.
Medium-size, circular registered groups tend to collect resource at a greater per-cell rates than large, small, ovular, or irregularly distributed groups.
In order to promote group turnover, we counteract the advantage of established group with a simple aging scheme.
As registered groups age over elapsed updates and elapsed somatic generations, their constituent cells lose the ability regenerate somatic tissue and then, soon after, to collect resource

A complete description of the mechanisms behind these collective resource-collection and group aging mechanisms appears in supplementary materials \ref{sup:resource_collection_process} \ref{sup:channel_group_life_cycle}.

Because registered groups only systematically arise through inheritance, they imply a close hereditary relationship in addition to a potentially functionally cooperative relationship.
In this work, we screen for fraternal transitions in individuality with respect to these registered groups by evaluating resource sharing, antagonistic cell proliferation, and apoptosis across within-group and between-group contexts.

\subsection{Cell-Level Organisms}

Our experiments use cell-level digital organisms controlled by evolving
genetic programs.
We employ the SignalGP representation, which expresses function-like modules of linear program instruction sequences in response to stimuli.
This event-driven paradigm facilitates the evolution of dynamic interactions between digital organisms and their environment (including other digital organisms) \citep{lalejini2018evolving}.

Previous work evolving digital organisms in grid-based problem domains has relied on a single computational instance which designates a direction to act in via an explicit cardinal ``facing'' state or output \citep{goldsby2014evolutionary, goldsby2018serendipitous, grabowski2010early, biswas2014causes, lalejini2018evolving}.
We introduce novel methodology to facilitate the evolution of directionally-symmetric phenotypes which effectively incorporate directionally-specific state.
In this work, each cell employs four instances of SignalGP hardware: one ``facing'' each cardinal direction.
These computational instances all execute the same SignalGP program but are otherwise decoupled. % and may follow independent chains of execution and develop independent computational state
Instances within a cell may coordinate via so-called intracellular messaging.
Supplementary Figure \ref{fig:dishtinygp-cartoon} schematically depicts the configuration of the four SignalGP instances that constitute a single cell.
% in relation to the instances of neighboring cells that may receive extracellular messages from the focal cell

\subsection{Treatments}

We screened evolutionary replicates conducted under combinations of two experimental conditions:
\begin{enumerate}
\item flat versus hierarchically-nested registered groups and
\item group-mediated versus group-independent resource collection.
\end{enumerate}

The first experimental manipulation explores the effects of hierarchical nesting of kin-sensing and/or functional cooperation.
The second manipulation explores the effects of functional cooperation.

We mix and match these experimental manipulations in four treatments:
\begin{enumerate}
\item one flat registered group level with group-independent resource collection (``Flat-Even''; in-browser simulation \url{https://mmore500.com/hopto/i}),
\item one flat registered group level with group-mediated resource collection (``Flat-Wave''; in-browser simulation \url{https://mmore500.com/hopto/j}),
\item two hierarchically-nested registered group levels levels with group-independent resource collection (``Nested-Even''; in-browser simulation \url{https://mmore500.com/hopto/k}), and
\item two hierarchically-nested registered group levels with group-mediated resource collection (``Nested-Wave''; in-browser simulation \url{https://mmore500.com/hopto/l}).
\end{enumerate}
