\section{Supplementary Material}

\subsection{Resource Collection Process} \label{sup:resource_collection_process}

Resource appears at a single point then spreads outwards update-by-update in a diamond-shaped wave. The expanding wave halts at a predefined limit.
Cells must enter an ``activated'' state to harvest resource as it passes overhead.
The cell at the starting position of a resource wave is automatically activated, and will propagate the activation signal to neighboring cells on the same signaling channel.
The newly activated cells, in turn, activate their own neighbors registered to the same signaling channel.
Neighbors registered to other signaling channels do not activate.
Each cell, after sending the activation signal, enters a temporary quiescent state.
In this manner, cells sharing a signaling channel track and harvest an expanding resource wave.
The rate of resource collection for a cell is determined by the size and shape of of its same-channel signaling network;
small or fragmented same-channel signaling networks will frequently miss out on resource as it passes by.

Resource waves have a limited extent.
Cells that activate outside the extent of a resource wave collect no resource.
A long quiescent period ensures that erroneously activated cells miss several subsequent opportunities to collect resource and therefore will tend to collect resource at a slower rate.
In this manner, ``Goldilocks'' --- not to small and not too big --- signaling networks enjoy superior fitness.

Resource wave starting points (seeds) are tiled over the toroidal grid from a randomly chosen starting location such that the extents of the resource waves do not overlap.
All resource waves begin and proceed synchronously;
when they complete, the next resource waves are seeded.
This process provides efficient and spatially-uniform selection for ``Goldilocks'' same-channel signaling networks.

Cells control the size and shape of their same-channel signaling group through strategic reproduction.
Three choices are afforded: whether to reproduce at all, where among the four adjoining tiles of the toroidal grid to place their offspring, and whether the offspring should be registered to the parent's signaling channel or be given a random channel ID (in the range 1 to $2^{64} - 1$).
The probability of channel collision is miniscule: $60 \times 60 \times 2^{20}$ (the grid dimensions times the number of simulation updates) independent channel values will collide with probability less than $1 \times 10^{-9}$.
No guarantees are made about the uniqueness of a newly-generated channel ID, but chance collisions are rare.

In addition to ``signaling channel''-based resource collection, we provide a uniform inflow of $+0.0051$, sufficient for one reproduction approximately every thousand updates.

\subsection{Hierarchical Nesting} \label{sup:hierarchical_nesting}

Hierarchical levels are introduced into the system through multiple separate, but overlaid, instantiations of this resource wave/channel-signaling scheme.
We refer to each independent resource wave/channel-signaling system as a ``level.''
In some experimental treatments, we allowed two resource wave/channel-signaling levels, identified here as level one and level two.
On level one, resource waves extended a radius of two toroidal tiles.
On level two they extended a radius of six toroidal tiles.
On both levels, activated cells netted $+0.2$ resource from a resource wave, but did not collect any resource outside the extent of the resource wave.
Due to the different radii of resource waves on different levels, level one selects for small same-channel signaling networks and level two selects for large same-channel signaling networks.

Each cell contained a pair of separate channel IDs, the first for level one and the second for level two.
We kept these channel IDs hierarchically nested by constraining inheritance during reproduction.
Daughter cells could not inherit just the level-one channel ID, they could either
\begin{enumerate}
\item inherit both level-one and level-two channel ID,
\item inherit level-two channel ID but not level-one channel ID, or
\item inherit neither channel ID.
\end{enumerate}
Hierarchically nested channel IDs are analogous to a strict corporate organizational structure: all employees (i.e., cells) are members of one department (i.e., level-one channel network) and one corporation (i.e., level-two channel network) but no employee can be a member of two departments and no department can be a member of two corporations.
Figure \ref{fig:morphology-wt} depicts hierarchically nested channel states assumed by an evolved strain.

An evolutionary transition in individuality can readily be evaluated within the DISHTINY framework with respect to same-channel network groups.
In addition to a potentially functionally cooperative relationship, shared channel IDs --- which may only systematically arise through inheritance --- imply a close hereditary relationship.
Because new channel IDs arise first in a single cell, same-channel signaling networks are reproductively bottlenecked analogously to a "Staying Together" life cycle (rather than a "Coming Together" life cycle) \cite{staps2019emergence}.
This precludes chimeric groups, except for mutations arising from somatic reproduction and rare cases of channel ID collision.

To recognize an evolutionary transition in individuality, we can evaluate
\begin{enumerate}
\item whether cells with the same channel ID cooperate altruistically by assessing, for example, resource sharing, and
\item whether division of reproductive labor arises by assessing whether interior cells cede reproduction to those at the periphery.
\end{enumerate}
If cells sharing the same level-one channel fulfill these conditions, we would suppose that a first-level transition in individuality had occurred.
Likewise, if cells sharing the sharing the same level-two channel fulfill these conditions, we would suppose that a second-level transition in individuality had occurred.
Further, we can screen for the evolution of complex multicellularity by assessing cell-cell messaging, regulatory patterning, and functional differentiation between cells within the a same-channel signaling network \cite{knoll2011multiple}.

\subsection{Channel Group Life Cycle} \label{sup:channel_group_life_cycle}

Mature same-channel resource collecting groups enjoy a considerable advantage over fledging propagules.
Because of the isometric scaling relationship between surface area and perimeter, cooperative same-channel resource collecting groups can marshal more resource at their periphery.
In addition, because of their greater surface area, mature same-channel resource collecting groups are able to seed resource-wave events and collect resource at a higher per-cell rate.

In order to ensure channel group turnover and facilitate channel group propagation, we impose a timed phase-out of somatic reproduction and resource wave harvests.
For each cell, we track a channel generation counter at each resource wave level.
At the genesis of a new channel group, these counters are set to zero.
Daughter cells that expand a channel group's soma are initialized to a counter value one greater than their parent.
Additionally, all channel generation counters are incremented every 512 updates to ensure that soma ages even in the absence of reproduction.
When a cell's channel generation counter reaches 1.5 times the wave radius of its level, it can no longer produce somatic daughter cells.
Then, after two additional counter steps, cells lose their ability to seed resource wave events and collect resource.
Thus, as channel groups age over time, their constituent cells lose the ability regenerate somatic tissue and then, soon after, to collect resource.
To prevent complete stagnation in the case where all cells' channel generation counters expire we provide a uniform inflow of $+0.0051$, sufficient for one reproduction approximately every thousand updates.

Interaction between nested channel groups produces a notable selective byproduct.
Because smaller, level-one channel groups tend to have intrinsically shorter lifespans, in order to achieve the full potential productive somatic lifespan of a larger, level-two channel group its constituent small channel groups must be intermittently regenerated.
Otherwise, the soma's capacity to seed resource-wave events and to collect resource will be prematurely lost once its constituent smaller, level-one channel groups expire.

This aging scheme's design ultimately stems from a desire
\begin{enumerate}
\item to facilitate evolution through regular turnover of emergent individuals and
\item to scaffold workable propagation for primitive cellular strategies while furnishing opportunities for more sophisticated adaptations to the imposed life cycle constraints.
\end{enumerate}
However, in some sense the aging scheme is heavy-handed, in effect enforcing rather than enabling a birth-death life cycle.
The evolutionary basis of aging and mortality --- in particular, the possibility of intrinsic evolutionary adaptations promoting these phenomena in addition to extrinsic factors  --- remains an active topic of scientific discussion \cite{baig2014evolution}.
In future work, we are interested in evaluating the outcomes of relaxing constraints of this aging scheme under different evolutionary conditions (such as cosmic ray mutations or irregular population structure) in light of theory attributing mortality and aging to evolvability, mutational accumulation, and costly somatic maintenance.

\subsection{Cell-Level Organisms} \label{sup:cell_level_organisms}

SignalGP programs are collections of independent procedural functions, each equipped with a bit-string tag \cite{lalejini2018evolving}.
A function is triggered by a signal with affinity that maximally and sufficiently matches its tag.
(A binding threshold of 0.1 was used in these experiments).
Signals may be generated by the environment, received as messages from other agents, or triggered internally by function execution.
Signals, and the ensuing chains of procedural execution they give rise to, are processed pseudo-concurrently by 24 virtual CPU cores.
Figure \ref{fig:signalgp-cartoon} schematically depicts a single SignalGP instance.

In this work, we introduce a regulatory extension to the SignalGP system.
During runtime, instructions may increase or decrease each tagged function's intrinsic tendency to match with --- and activate in response to --- tagged queries.
Intrinsic tag-to-tag match distances $m$ are modulated by a regulator value $r$ (baseline, 1.0) to become $r + r \times m$.
This scheme allows a function to be upregulated such that every query activates that function (e.g., $r = 0$) or no query activates that function (e.g., $r = \texttt{inf}$).
These regulation settings are heritable during reproduction but automatically decay after a number of updates determined when they are set.

To allow cells to protect themselves form potentially antagonistic interactions with their neighbors, we filter intercellular messages through a tag-matching membrane.
At runtime, cells can embed tags in this membrane that either admit or repel incoming messages.
Messages that do not match with a membrane tag are repelled.
A message, for example, that would activate a SignalGP function containing an apoptosis instruction could be rejected while other messages are accepted.
Tags embedded in this membrane automatically decay and may also be regulated.
We also filter messages between hardware instances within the same cell through a tag-matching membrane, but the default behavior for messages with unmatched tags is admission rather than rejection.

Previous work evolving digital organisms in grid-based problem domains has relied on a single computational instance which designates a direction to act in via an explicit cardinal ``facing'' state or output \cite{goldsby2014evolutionary, goldsby2018serendipitous, grabowski2010early, biswas2014causes, lalejini2018evolving}.
Under this paradigm, a large portion of genotype space encodes behaviors that are intrinsically asymmetrical with respect to absolute or relative (depending on implementation) cardinal direction.
However, in grid-based tasks, directional phenotypic symmetry is generally advantageous.
That is --- in the absence of a polarizing external stimulus --- successful agents generally behave uniformly with respect to each cardinal direction of the grid.
In this work, each cell employs four instances of SignalGP hardware: one ``facing'' each cardinal direction.
These computational instances all execute the same SignalGP program but are otherwise decoupled and may follow independent chains of execution and develop independent regulatory states.
Instances within a cell execute round robin step-by-step in an order that is randomly drawn at the outset of each update.

Genetic encodings that exploit problem-domain symmetries are known to promote evolvability and --- ultimately --- evolved solution quality \cite{clune2011performance, cheney2014unshackling}.
We submit that this directional hardware replication protocol likely increases the fraction of genotype space that encodes cardinally-symmetric phenotypes and therefore better facilitates the evolution of high-fitness phenotypes.
In further work, we look forward to exploring the evolvability and solution quality implications of this new approach.

The single SignalGP program that is mirrored across the cell's computational instances represents the cell's genome.
Mutation, with standard SignalGP mutation parameters as in \cite{lalejini2018evolving}, is applied to 1\% of daughter cells at birth.
In addition, genomes encode the bitstrings associated with environmental events.
These bitstrings evolve at a per-bit mutation rate equivalent to the bitstring labels of SignalGP functions.

Instances within a cell may send intracellular messages to one another or intercellular messages to a neighboring cell.
Intercellular messages are received by the SignalGP instance that faces the sending cell.
Figure \ref{fig:dishtinygp-cartoon} schematically depicts the configuration of the four SignalGP instances that constitute a single DISHTINY cell as well as the instances of neighboring cells that receive extracellular messages from the focal cell.

\subsection{Standard SignalGP Instruction Library} \label{sup:standard_instruction_library}

The default SignalGP instruction set defines a number of generic arithmetic, logic, utility, and program flow instructions \cite{lalejini2018evolving}.
We include these instructions in our experiment's instruction library.

To counteract crowding of the mutational landscape by the volume of custom instructions provided, a second identical copy of each standard SignalGP instruction was included in the library.

\begin{itemize}
\item \textbf{Increment}
Increment value in a designated register.
\item \textbf{Decrement}
Decrement value in a designated register.
\item \textbf{Not}
Logically toggle value in a designated register.
\item \textbf{Add}
Add values from two designated registers into a third designated register.
\item \textbf{Subtract}
Subtract values from two designated registers into a third designated register.
\item \textbf{Subtract}
Subtract values from two designated registers into a third designated register.
\item \textbf{Multiply}
Multiply values from two designated registers into a third designated register.
\item \textbf{Divide}
Divide values from two designated registers into a third designated register.
\item \textbf{Modulus}
Calculate the modulus from two designated registers and place result into a third designated register.
\item \textbf{Test Equal}
Compare values in two designated registers and place equality result into a third designated register.
\item \textbf{Test Non-equality}
Compare values in two designated registers and place opposite equality result into a third designated register.
\item \textbf{Test Less}
Compare values in two designated registers and place less-than result into a third designated register.
\item \textbf{If}
If a designated register is non-zero, proceed.
Otherwise, skip block.
\item \textbf{While}
While a designated register is non-zero, loop over a program block.
Otherwise, skip block.
\item \textbf{Countdown}
While a designated register is non-zero, loop over a program block and decrement the value in the designated register.
Otherwise, skip block.
\item \textbf{Close}
If a preceding program block is, close it.
\item \textbf{Break}
Break to the end of the current program block.
\item \textbf{Call}
Call the SignalGP program module that best matches instruction's affinity.
\item \textbf{Return}
If possible, return from the current function.
\item \textbf{Set Memory}
Set a designated register's value to hard-coded memory value.
\item \textbf{Set True}
Set a designated register's value to true (1.0).
% terminal
\item \textbf{Copy Memory}
Copy the value of a designated register to a second designated register.
\item \textbf{Swap Memory}
Swap the values of two designated registers.
\item \textbf{Input}
Copy a designated element of input memory into a designated register.
\item \textbf{Output}
Copy to a designated element of output memory from a designated register.
\item \textbf{Commit}
Copy a designated register into a designated element of global memory.
\item \textbf{Pull}
Copy a designated element of global memory into a designated register.
\item \textbf{Fork}
Fork a new thread with the SignalGP program module that best matches the instruction's affinity.
\item \textbf{Terminate}
Terminate the current thread.
\item \textbf{Nop}
No operation.
\item \textbf{RNG Draw}
Draw a random value between 0.0 and 1.0 from random number generator and store result in a register.
\item \textbf{Set Regulator}
Set the program module regulator that best matches (without regulation) the instruction's affinity to the value of a designated register.
\item \textbf{Set Own Regulator}
Set the program module regulator of the currently-executing program module to value of a designated register.
\item \textbf{Adjust Regulator}
Adjust the program module regulator of the program module that best matches (without regulation) the instruction's affinity a designated fraction toward a designated register's value.
\item \textbf{Adjust Own Regulator}
Adjust the program module regulator of the currently-executing program module a designated fraction toward a designated register's value.
\item \textbf{Extend Regulator}
Adjust the program module regulator decay timer of the program module that best matches (without regulation) the instruction's by a designated register's value.
\item \textbf{Sense Regulator}
Copy the program module regulator value of the program module that best matches (without regulation) the instruction's affinity into a designated register.
\item \textbf{Sense Own Regulator}
Copy the program module regulator value of currently-executing program module into a designated register.
\end{itemize}

\subsection{Custom Instruction Library} \label{sup:custom_instruction_library}

We define a number of custom instructions to allow evolving programs to sense and interact with their environments, including
\begin{itemize}
\item reproduction,
\item resource sharing,
\item channel ID sensing,
\item apoptosis,
\item intracellular messaging, and
\item intercellular messaging.
\end{itemize}

We provide an listing of our experiment's instruction library below.

Instructions that involve an extracellular neighbor default to the cell that the executing SignalGP instance is facing.
To ensure a founding crop of viable individuals, apoptosis and program flow instructions in the initial randomly-generated population were replaced with no-op instructions.
However, these instructions were allowed to mutate in to genomes freely once evolutionary runs began.

\begin{itemize}
\item \textbf{Send Intracellular Message}
Send a message to a single other SignalGP instance within the cell specified by a designated register's value.
\item \textbf{Broadcast Intracellular Message}
Send a message to all SignalGP instances within the cell, excluding self.
\item \textbf{Put Internal Membrane Gatekeeper}
% bringer / blocker
Place a tag in the internal membrane that, depending on insertion order, admits or blocks incoming internal messages it matches with.
\item \textbf{Send Intercellular Message}
Send a message to a single cellular neighbor.
\item \textbf{Broadcast Intercellular Message}
Send a message to all cellular neighbors.
\item \textbf{Put External Membrane Gatekeeper}
% bringer / blocker
Place a tag in the external membrane that, depending on insertion order, admits or blocks incoming external messages it matches with.
\item \textbf{Set External Membrane Regulator}
Set the regulation of the gatekeeper in the external membrane that best matches (without regulation) the instruction's affinity to the value of a designated register.
\item \textbf{Adjust External Membrane Regulator}
Adjust the regulation of the gatekeeper in the external membrane that best matches (without regulation) the instruction's affinity a designated fraction toward a designated register's value.
\item \textbf{Sense External Membrane Regulator}
Copy the regulation value of the program module that best matches (without regulation) the instruction's affinity into a designated register.
\item \textbf{Activate Intercellular Inbox}
Mark the intercellular inbox to accept messages.
At cell birth, the inbox is deactivated.
\item \textbf{Deactivate Intercellular Inbox}
Mark the intercellular inbox to decline messages.
\item \textbf{Share Resource}
Send a proportion of the cell's stockpiled resource to a neighboring cell.
One instruction defaults to sending a large proportion of available resource (50\%) to the neighboring cell.
A second instruction defaults to sending a small proportion of available resource (5\%) to the neighboring cell.
The proportion of available resource can be adjusted by a register-based argument.
\item \textbf{Set Stockpile Sharing Reserve}
Designate a quantity of stockpiled resource as ineligible for sharing.
The amount may be modified by a register-based argument.
\item \textbf{Clear Stockpile Sharing Reserve}
Designate all stockpiled resource as eligble for sharing.
\item \textbf{Restrict Outgoing Shared Resource}
Reduce outgoing sharing efficacy.
Unsent resource is retained by the sending cell (with no resource lost).
The fraction reduced is determined by a register-based argument.
\item \textbf{Restrict Incoming Shared Resource}
Reduce incoming sharing efficacy.
Declined resource is retained by the sending cell (with no resource lost).
The fraction reduced is determined by a register-based argument.
\item \textbf{Reproduce}
Attempt to spawn a child cell in a particular direction, paid for out of the parent cell's resource stockpile.
If sufficient resource is not available in the cell's stockpile, no resource is action is taken.
Variants of this instruction are defined for each channel ID inheritance level: from endowing the daughter cell with the parental channel IDs across all levels, to endowing the daughter cell with a new level-one channel ID but the parent's level-two channel ID, to endowing the daughter cell with all-new channel IDs.
If a channel generation counter limit has been reached, reproduction is simply attempted at the next highest level; even with channel generation counters maxed out, cells may generate offspring with all-new channel IDs.
\item \textbf{Pause Reproduction}
Pause cellular reproduction in a single direction for the remainder of the current update and for the entire next update.
Variants of this instruction pause reproduction at a certain wave/channel-signaling level or across all channel ID inheritance levels.
\item \textbf{Set Stockpile Reproduction Reserve}
Designate a quantity of stockpiled resource as ineligible for use to reproduce.
The amount may be modified by a register-based argument.
\item \textbf{Clear Stockpile Sharing Reserve}
Designate all stockpiled resource as eligible for use to reproduce.
\item \textbf{Apoptosis}
The cell is killed at the end of the current update.
\item \textbf{Designate/Revoke Heir} A dying cell's own stockpile is split evenly among neighboring cells that are designated at the time of death.
On apoptosis, 50\% of the reproduction cost to establish a cell is also split between designated neighboring cells.
These instructions mark or un-mark a neighbor as a heir.
\item \textbf{Increase Channel Generation Counter}
Increases the cell's channel generation counter.
The amount the cell's generation counter is increased by can be adjusted by register-based argument.
\item \textbf{Query Own Stockpile}
Sets a designated register to the amount of resource present in the cell's stockpile.
\item \textbf{Query Own Channel Generation Counter}
This instruction sets a designated register to the value of the cell's channel generation counter.
A variant of this instruction is provided for each wave/channel-signaling level.
\item \textbf{Query ``Is Neighbor Live?''}
This instruction sets a designated register to 1 if the neighboring tile contains a live cell and 0 otherwise.
\item \textbf{Query ``Is Neighbor My Cellular Child?''}
This instruction sets a designated register to 1 if the neighboring cell is the daughter of the querying cell and 0 otherwise.
\item \textbf{Query ``Is Neighbor My Cellular Parent?''}
This instruction sets a designated register to 1 if the neighboring cell is the parent of the querying cell and 0 otherwise.
\item \textbf{Query ``Does Neighbor's Channel ID Match Mine?''}
This instruction sets a designated register to 1 if the neighboring cell has the same channel ID as the querying cell and 0 otherwise.
A variant of this instruction is provided for each wave/channel-signaling level.
\item \textbf{Query ``Does Neighbor's Channel ID Descend From Mine?''}
This instruction sets a designated register to 1 if the neighboring cell's highest-level channel ID is different from the querying cell's highest-level channel ID, but is descended from the querying cell's channel ID via an explicit propagule-generating reproduction call.
This instruction allows a querying cell to sense whether its neighbor is a member of a same-channel group that is a propagule of the querying cell's same-channel group.
\item \textbf{Query ``Does My Channel ID Descend From Neighbor's?''}
This instruction sets a designated register to 1 if the querying cell's highest-level channel ID is different from the neighboring cell's highest-level channel ID, but is descended from the neighboring cell's channel ID  via an explicit propagule-generating reproduction call.
This instruction allows a querying cell to sense whether it is a member of a same-channel group that is a propagule of the neighboring cell's same-channel group.
\item \textbf{Query ``Is Neighbor Poorer?''}
This instruction sets a designated register to 1 if the querying cell's resource stockpile is larger than the neighboring cell's.
\item \textbf{Query ``Is Neighbor Older?''}
This instruction sets a designated register to 1 if the querying cell's cell age is less than the neighboring cell's.
\item \textbf{Query ``Is Neighbor Expired?''}
This instruction sets a designated register to 1 if a neighboring cell's channel generation counter has exceeded the expiration threshold.
\item \textbf{Query Neighbor's Channel ID}
This instruction sets a designated register to the neighbor's channel ID.
A variant of this instruction is provided for each wave/channel-signaling level.
\item \textbf{Query Neighbor's Stockpile}
This instruction sets a designated register to the amount of resource present in the neighbor's stockpile.
\end{itemize}

\subsection{Environmental Cue Library} \label{sup:environmental_cue_library}

Event-driven sensing has been shown to enable evolution of SignalGP programs that more successfully react to  environmental state \cite{lalejini2018evolving}, so we supplement our instruction-based sensors with event-based input.
Every eight updates, a subset of environmental events are triggered on each SignalGP hardware based on current local environmental conditions.
The activating affinity of each event is genetically-encoded as part of the program currently executing on the hardware.
We provide a listing of our experiment's event library in supplementary material \ref{sup:environmental_cue_library}.

\begin{itemize}
\item \textbf{On Update}
This event is triggered every eight updates.
\item \textbf{Just Born}
This event is triggered once after a cell is born.
\item \textbf{Richer Neighbor}
This event is triggered if a neighbor cell has more stockpiled resource than the focal cell.
\item \textbf{Poorer Neighbor}
This event is triggered if a neighbor cell has less stockpiled resource than the focal cell.
\item \textbf{Facing Cellular Child}
This event is triggered if the SignalGP instance is facing a neighboring cell that is the querying cell's daughter.
\item \textbf{Facing Cellular Parent}
This event is triggered if the SignalGP instance is facing a neighboring cell that is the querying cell's parent.
\item \textbf{Neighbor's Channel ID Descends From Mine}
This event is triggered if the neighboring cell's highest-level channel ID is different from the querying cell's highest-level channel ID, but is descended from the querying cell's channel ID via an explicit propagule-generating reproduction call.
This event allows a querying cell to sense whether its neighbor is a member of a same-channel group that is a propagule of the querying cell's same-channel group.
\item \textbf{My Channel ID Descends From Neighbor's}
This event is triggered if the focal cell's highest-level channel ID is different from the neighboring cell's highest-level channel ID, but is descended from the neighboring cell's channel ID via an explicit propagule-generating reproduction call.
This event allows a neighboring cell to sense whether its neighbor is a member of a same-channel group that is a propagule of the neighboring cell's same-channel group.
\item \textbf{Neighbor's Channel ID Matches Mine}
This event is triggered if a SignalGP instance is facing a neighbor cell that shares its channel ID.
A different event is provided for each resource wave/channel-signaling level.
\item \textbf{Neighbor's Channel ID Does Not Match Mine}
This event is triggered if a SignalGP instance is facing a neighbor cell that does not share its channel ID.
A different event is provided for each resource wave/channel-signaling level.
\item \textbf{Channel Generation Counter Is Unexpired}
This event is triggered if a SignalGP instance's cell's channel generation counter has not yet reached the expiration threshold.
A different event is provided for each resource wave/channel-signaling level.
\item \textbf{Channel Generation Counter Is Expiring}
This event is triggered if a SignalGP instance's cell's channel generation counter has not yet reached the threshold where somatic propagation capacity, but not resource accumulation capacity, is lost.
A different event is provided for each resource wave/channel-signaling level.
\item \textbf{Channel Generation Counter Is Expired}
This event is triggered if a SignalGP instance's cell's channel generation counter has not yet reached the threshold where both somatic propagation capacity and resource accumulation capacity are lost.
A different event is provided for each resource wave/channel-signaling level.
\item \textbf{No-reward Resource Activation}
This event is triggered if a SignalGP instance's cell experiences a resource collection activation where no resource reward is achieved (e.g., the cell lies extent of the resource wave).
A different event is provided for each resource wave/channel-signaling level.
\end{itemize}

\subsection{Treatments} \label{sup:treatments}

\begin{table*}[!htbp]
\begin{center}

\begin{filecontents*}{productivity.csv}
Measure,Nested,Flat,Even
Per-cell-update resource inflow,$0.0178 \pm 0.0010$,$0.0145 \pm 0.0008$,$0.0175 \pm 0$
Per-cell-update cell reproduction,$0.0117 \pm 0.0023$,$0.0094 \pm 0.0018$,$0.0113 \pm 0.0018$
\end{filecontents*}

\begin{tabular}{l|c|c|c}%
\bfseries Measure
  & \bfseries Even
  & \bfseries Flat
  & \bfseries Nested
\csvreader[head to column names]{productivity.csv}{}
{\\\hline\Measure
  & \Nested
  & \Flat
  & \Even
}
\end{tabular}

\caption{
Observed productivity at epoch 1 (mean $\pm$ S.D.)
}
\label{tab:productivity}
\end{center}
\end{table*}


\begin{figure*}[!htbp]
\begin{center}

\begin{filecontents*}{systematics.csv}
Measure,NestedShort,FlatShort,EvenShort,NestedLong,FlatLong,EvenLong
Replicate count,40,40,40,37,40,39
Cellular generations elapsed,$953 \pm 563$,$1140 \pm 697$,$935 \pm 636$,$4635 \pm 3077$,$5869 \pm 3246$,$4628 \pm 3274$
Level 1 generations elapsed,$566 \pm  637$,$127 \pm 68$,$592 \pm 553$,$2637 \pm 2933$,$654 \pm 353$,$2908 \pm 2768$
Level 2 generations elapsed,$121 \pm  85$,N/A,$117 \pm  99$,$546 \pm 349$,N/A,$526 \pm  352$
Phylogenetic depth,$14 \pm 6$,$14 \pm  6$,$15 \pm 8$,$54 \pm 31$,$55 \pm 26$,$60 \pm 44$
Coalescent replicates,58.5\%,77.5\%,62.5\%,94.6\%,97.5\%,92.3\%
\end{filecontents*}

\begin{tabular}{l|c|c|c|c|c|c}%
&\multicolumn{3}{c|}{Update 262144}
&\multicolumn{3}{c}{Update 1048576}\\
\bfseries Measure
  & \bfseries Nested
  & \bfseries Flat
  & \bfseries Even
  & \bfseries Nested
  & \bfseries Flat
  & \bfseries Even
\csvreader[head to column names]{systematics.csv}{}
{\\\hline\Measure
  & \NestedShort
  & \FlatShort
  & \EvenShort
  & \NestedLong
  & \FlatLong
  & \EvenLong
}
\end{tabular}

\caption{
Systematics information TODO
}
\label{fig:systematics}
\end{center}
\end{figure*}


In this work, we screened replicates conducted under combinations of two experimental conditions:
\begin{enumerate}
\item flat versus nested hierarchical resource wave/channel-signaling levels and
\item cooperative versus independent resource collection.
\end{enumerate}

The first experimental manipulation explores the effects of hierarchical nesting of kin-sensing and/or functional cooperation.
The second manipulation explores the effects of functional cooperation.

To enact the first manipulation, we compared the nested hierarchical resource wave/channel-signaling scheme described above with a single-level scheme with waves extending six toroidal tiles.
We also increased the resource wave reward to $+0.6$ to approximately match the observed resource inflow rate of the nested scheme.
To enact the second manipulation, we removed the resource wave reward and increased the uniform resource inflow rate to $+0.0175$ in order to approximately match the net inflow rate under the dual-level wave-based scheme.
Table \ref{tab:productivity} reports productivity observed under these different conditions.

We mix and match these experimental manipulations in three treatments:
\begin{enumerate}
\item one level with even resource (``Flat-Even''; in-browser simulation \url{https://mmore500.com/hopto/i}),
\item one level with wave-based resource (``Flat-Wave''; in-browser simulation \url{https://mmore500.com/hopto/j}),
\item two levels with even resource (``Nested-Even''; in-browser simulation \url{https://mmore500.com/hopto/k}), and
\item two levels with wave-based resource (``Nested-Wave''; in-browser simulation \url{https://mmore500.com/hopto/l}).
\end{enumerate}

We ran 40 replicates under each treatment condition.
Replicates were seeded with randomly generated SignalGP programs.
To conserve disk space, we divided evolutionary runs into 262144 ($2^{18}$) update epochs and collected data in 8096 ($2^{13}$) update snapshots between epochs.
All replicates ran at least one full epoch, and all comparisons between or within treatments are conducted at this time point.
However, most replicates (156/160) were able to run to four epochs during available compute time.
We screened for and conducted case studies at the latest available data for each replicate.
All reported case studies happen to be drawn from runs that completed 4 epochs of evolution.
Table \ref{tab:systematics} reports the systematics outcomes observed under each treatment at epoch 1 and at epoch 4.

All experiments took place a traditional 60-by-60 toroidal grid, supporting a population of at most 3600 individual cells.

\subsection{Competition Experiments and Phenotype Assays} \label{sup:competition_assays}

We performed further experiments to develop case studies of evolved strains we manually screened from our evolutionary runs.
In these experiments, the most-abundant genotype was harvested from the end-state of evolutionary runs as the wild type strain.
We collected epigenetic state (i.e., regulatory settings) along with genetic state (i.e., SignalGP program and environmental-cue-to-tag mapping).
All further work with harvested strains was conducted under environmental conditions identical to that of the treatment they evolved in.

To analyze the relative fitness of knockout strains versus wild type, we seeded 20 $60 \times 60$ toroidal grids with ten cells of each strain, including epigenetic regulator state.
We ran competition experiments for the duration of one snapshot.
Seeded cells generally proliferated to completely fill the toroidal grid in the first quarter of the snapshot.
Competition experiment outcomes were determined by strains' relative cell populations within the grid at the end of the snapshot.

To perform phenotypic comparisons between knockout strains and wild type, we seeded ten cells of each strain onto separate $60 \times 60$ toroidal grids and then cultured them for the duration of one snapshot.

\subsection{Implementation} \label{sup:implementation}

We implemented our experimental system using the Empirical library for scientific software development in C++, available at \url{https://github.com/devosoft/Empirical}.
The code used to perform and analyze our experiments, our figures, data from our experiments, and a live in-browser demo of our system is available via the Open Science Framework at \url{https://osf.io/g58xk/}.
Most replicates finished within a day, but some took up to a week to complete.
